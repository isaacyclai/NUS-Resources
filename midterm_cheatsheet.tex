\documentclass[frenchspacing,9pt,landscape,a4paper]{article}
\usepackage{amssymb,amsmath,amsthm,amsfonts}
\usepackage{multicol,multirow}
\usepackage{calc}
\usepackage{ifthen}
\usepackage[landscape]{geometry}
\usepackage[colorlinks=true,citecolor=blue,linkcolor=blue]{hyperref}
\usepackage[linewidth=1pt]{mdframed}
\usepackage{enumitem}
\usepackage{times} % Times New Roman font to fit more content

% Even more efficient font
\usepackage[bitstream-charter]{mathdesign}
\usepackage[T1]{fontenc}

\usepackage{graphicx}

\setlist[itemize]{align=parleft,left=0pt..1em}
\setlist[enumerate]{align=parleft,left=0pt..1em}

\ifthenelse{\lengthtest { \paperwidth = 11in}}
    { \geometry{top=.5in,left=.5in,right=.5in,bottom=.5in} }
	{\ifthenelse{ \lengthtest{ \paperwidth = 297mm}}
		{\geometry{top=0.8cm,left=0.8cm,right=0.8cm,bottom=0.8cm} }
		{\geometry{top=0.8cm,left=0.8cm,right=0.8cm,bottom=0.8cm} }
	}
\pagestyle{empty}
\makeatletter
\renewcommand{\section}{\@startsection{section}{1}{0mm}%
                                {-1ex plus -.5ex minus -.2ex}%
                                {0.5ex plus .2ex}%x
                                {\normalfont\large\bfseries}}
\renewcommand{\subsection}{\@startsection{subsection}{2}{0mm}%
                                {-1explus -.5ex minus -.2ex}%
                                {0.5ex plus .2ex}%
                                {\normalfont\normalsize\bfseries}}
\renewcommand{\subsubsection}{\@startsection{subsubsection}{3}{0mm}%
                                {-1ex plus -.5ex minus -.2ex}%
                                {1ex plus .2ex}%
                                {\normalfont\small\bfseries}}
\makeatother
\setcounter{secnumdepth}{0}
\setlength{\parindent}{0pt}
\setlength{\parskip}{0pt plus 0.5ex}
\setlength{\columnseprule}{0.1pt}

\newcommand{\BR}{\mathbb R}
\newcommand{\BC}{\mathbb C}
\newcommand{\BF}{\mathbb F}
\newcommand{\BQ}{\mathbb Q}
\newcommand{\BZ}{\mathbb Z}
\newcommand{\BN}{\mathbb N}

\newcommand{\mb}[1]{\mathbf #1}

\newcommand{\floor}[1]{\left\lfloor #1 \right\rfloor}
\newcommand{\ceiling}[1]{\left\lceil #1 \right\rceil}
\newcommand{\abs}[1]{\left\lvert #1 \right\rvert}
\newcommand{\norm}[1]{\left\lVert #1 \right\rVert}
\newcommand{\innerproduct}[2]{\left\langle #1, #2 \right\rangle}

\newcommand{\ddx}{\frac{d}{dx}}
\newcommand{\curl}{\text{curl }}
\newcommand{\dvg}{\text{div }}

\newcommand{\adj}{\text{adj }}

\DeclareMathOperator*{\argmax}{arg\,max}
\DeclareMathOperator*{\argmin}{arg\,min}
\DeclareMathOperator{\pr}{Pr}
\DeclareMathOperator{\var}{Var}

% -----------------------------------------------------------------------
\newtheorem*{thm}{Theorem}
\newtheorem*{defn}{Definition}
\title{MA4261 cheatsheet}

\begin{document}

\raggedright
\footnotesize

\begin{multicols}{5} % can work with 4 too
\setlength{\premulticols}{1pt}
\setlength{\postmulticols}{1pt}
\setlength{\multicolsep}{1pt}
\setlength{\columnsep}{2pt}
\begin{mdframed}
\begin{center}
    \large{\textbf{CS3231 Theory of Computation}} \\
    \normalsize{\textbf{AY24/25 Semester 1}}\\
    \small{by Isaac Lai}
\end{center}	
\end{mdframed}
\section{Finite Automata}
\subsection{DFAs}
\begin{defn}
    A DFA is a 5-tuple $(Q,\Sigma,\delta,q_0,F)$ where
    \begin{itemize}
        \item $Q$ is a finite set of states
        \item $\Sigma$ is a finite set of input symbols
        \item  $\delta(q,w)$ is a transition function that takes as input  $q\in Q$ and  $w\in\Sigma$
        \item  $q_0\in Q$ is a starting state
        \item $F\subseteq Q$ is the set of accepting states
    \end{itemize}
\end{defn}
\subsubsection{Properties}
\begin{itemize}
    \item We can extend $\delta$ as  $\hat{\delta}$ which accepts strings. Basis:
        $\hat{\delta}(q,\epsilon)=q$, Induction:  $\hat{\delta}(q,xa)=\delta(\hat{\delta}(q,x),a)$ 
    \item The \textbf{language} accepted by DFA $A$ is  $L(A)=\{w\mid\hat{\delta}(q_0,w)\in F\}$
    \item $q$ is a \textbf{dead state} if  $\forall w\in\Sigma^*$,  $\hat{\delta}(q,w)\notin F$, i.e. an
        accepting state cannot be reached from $q$
    \item  $q$ is an \textbf{unreachable state} if  $\forall w\in\Sigma^*$,  $\hat{\delta}(q_0,w)\neq q$,
        i.e. $q$ cannot be reached from the starting state
\end{itemize}
\subsection{NFAs}
\begin{defn}
    An NFA is a 5-tuple $(Q,\Sigma,\delta,q_0,F)$ defined similarly to a DFA except $\delta$ maps the input
    (state, symbol) to a set of states (i.e. a subset of $Q$)
\end{defn}
\subsubsection{Properties}
\begin{itemize}
    \item We can extend $\delta$ as  $\hat{\delta}$ which accepts strings. Basis:
        $\hat{\delta}(q,\epsilon)=\{q\}$, Induction:
        $\hat{\delta}(q,xa)=\bigcup_{p\in\hat{\delta}(q,x)}\delta(p,a)$
    \item The \textbf{language} accepted by an NFA is $L(A)=\{w\mid\hat{\delta}(q_0,w)\cap F\neq\emptyset\}$
    \item \textbf{NFA with $\epsilon$-transitions}: $\delta$ maps  $Q\times(\Sigma\cup\{\epsilon\})$ to
        subsets of $Q$
\end{itemize}
\begin{defn}[$\epsilon$ closures]
    \begin{enumerate}
        \item $q\in Eclose(q)$
        \item If state  $p\in Eclose(q)$, then each state in  $\delta(p,\epsilon)$ is in  $Eclose(q)$
        \item Iterate step 2, until no more changes can be done to  $Eclose(q)$
    \end{enumerate}
\end{defn}
\begin{itemize}
    \item For $\epsilon$-NFA, we can extend  $\delta$ to  $\hat{\delta}$ where
        $\hat{\delta}(q,\epsilon)=Eclose(q)$ and  $\hat{\delta}(q,wa)=\bigcup_{p\in R}Eclose(p)$ 
\end{itemize}
\subsection{Equivalence}
\begin{itemize}
    \item NFAs (and  $\epsilon$-NFAs) and DFAs are \textbf{equivalent}
    \item To convert an NFA to a DFA, create one state in the DFA for every subset of states in the NFA.
        Draw the transitions accordingly from the NFA for each state in the DFA (which is a subset of the
        original states)
\end{itemize}
\subsection{Minimisation of DFA}
Suppose we have a DFA $A=(Q,\Sigma,\delta,q_0,F)$. States $p,q$ are indistinguishable iff for all  $w$,
$\hat{\delta}(p,w)\in F$ iff  $\hat{\delta}(q,w)\in F$. To build a table to determine distinguishable
pairs,
\begin{enumerate}
    \item Base case: Initially, each $(p,q)$ pair such that  $p\in F$ and  $q\notin F$ (or vice versa) is
        distinguishable
    \item Inductive step: For any  $a\in\Sigma$, if  $\delta(p,a)$ and  $\delta(q,a)$ are distinguishable,
        then  $(p,q)$ are distinguishable
    \item Continue the inductive step, till no more pairs of distinguishable states can be added
\end{enumerate}
The new DFA is formed by
\begin{enumerate}
    \item First delete all non-reachable states
    \item Find all nondistinguishable pairs of states
    \item Each pair of non-distinguishable states is equivalent, and gives an equivalence relation
    \item States of the new DFA are these equivalence classes
    \item If $\delta(p,a)=q$ in the original DFA, then  $\delta_{new}(E_p,a)=Eq$ where  $E_p$  and $E_q$
        are equivalence classes corresponding to  $p$ and  $q$ respectively
    \item Initial state of the new DFA is the equivalence class containing the start state of the original
        DFA, final states of the new DFA are all equivalence classes containing a final state
\end{enumerate}
\subsection{Parallel Simulation of 2 DFAs}
\begin{itemize}
    \item Suppose we have $A=(Q,\Sigma,\delta,q_0,F)$ and $A'=(Q',\Sigma,\delta',q_0',F')$
    \item Let $A''=(Q\times Q',\Sigma,\delta',(q_0,q_0'),F'')$ where
        $\delta''((q,q'),a)=(\delta(q,a),\delta'(q',a))$ and  $F''$ depends on the need. Then  $A''$
        simulates  $A$ and  $A'$ in parallel 
    \item If $F''=F\times F'$, then  $A''$ accepts the intersection of languages accepted by  $A$ and  $A'$
    \item If  $F''=F\times Q'\cup Q\times F'$,  $A''$ accepts the union of languages accepted by  $A$
        and  $A'$
\end{itemize}
\section{Regular Languages}
\subsection{Basis}
\begin{itemize}
    \item $\epsilon$ and  $\emptyset$ are regular expressions,  $L(\epsilon)=\{\epsilon\}$ and
        $L(\emptyset)=\emptyset$
    \item If $a\in\Sigma$, then $a$ is a regular expression, and  $L(a)=\{a\}$
\end{itemize}
\subsection{Induction}
If $r_1,r_2$ are regular expressions, then so are 
\begin{itemize}
    \item $r_1+r_2$. The language is $L(r_1+r_2)=L(r_1)\cup L(r_2)$ 
    \item $r_1\cdot r_2$. The language is $L(r_1\cdot r_2)=\{xy\mid x\in L(r_1),y\in L(r_2)\}$
    \item $r_1^*$. The language is $L(r_1^*)=\{x_1x_2,\dots,x_k\mid 1\leq i\leq k,x_i\in L(r_1)\}$
    \item $(r_1)$. The language is $L((r_1))=L(r_1)$
\end{itemize}
\subsection{DFA to Regex}
\begin{itemize}
    \item $R_{i,j}^k$ denotes the regex of strings which can be formed by going from state  $i$ to  $j$
        using intermediate states  $\leq k$
    \item Base case: for  $R_{i,j}^0$
         \begin{itemize}
             \item If $i\neq j$,  $R_{i,j}^0=a_1+a_2+\cdots+a_m$ where $a_1,\dots,a_m$ are all symbols such
                 that $\delta(i,a_r)=j$
             \item  If $i=j$,  $R_{i,i}^0=\epsilon+a_1+\cdots+a_m$ where $a_1,\dots,a_m$ are all symbols
                 such that $\delta(i,a_r)=i$
        \end{itemize}
    \item Inductive case: $R_{i,j}^{k+1}=R_{i,j}^k+R_{i,k+1}^k(R_{k+1,k+1})^*R_{k+1,j}^k$
    \item Regex for the DFA is $\sum_{j\in F}R_{1,j}^n$
\end{itemize}
\subsection{Regex Properties}
\begin{itemize}
    \item Operator precedence: $*,\cdot,+$
    \item $M+N=N+M$
    \item  $L(M+N)=LM+LN$
    \item  $L+L=L$
    \item  $(L^*)^*=L^*$
    \item  $\emptyset^*=\epsilon$
    \item  $\epsilon^*=\epsilon$
    \item  $L^+=LL^*=L^*L$
    \item  $L^*=\epsilon+L^+$
    \item  $(L+M)^*=(L^*M^*)^*$
\end{itemize}
\subsection{Regular Language Properties}
\begin{thm}[Pumping Lemma]
    Let $L$ be a regular language. Then there exists a constant  $n$ (dependent on  $L$) such that for
    every string  $w\in L$ satisfying $\abs{w}\geq n$, $w$ can be broken into three strings  $w=xyz$, such
    that
     \begin{enumerate}
         \item $y\neq\epsilon$
         \item  $\abs{xy}\leq n$
         \item For all  $k\geq 0$,  $xy^kz\in L$
    \end{enumerate}
\end{thm}
\begin{itemize}
    \item If $L_1,L_2$ are regular, so is $L_1\cup L_2$
    \item If $L_1,L_2$ are regular, so is $L_1\cdot L_2$
    \item If $L$ is regular, so is  $\overline{L}=\Sigma^*-L$
    \item If $L_1,L_2$ are regular, so is $L_1\cap L_2$
    \item If $L_1,L_2$ are regular, so is $L_1-L_2$
    \item If $L$ is regular, so is  $L^R$
    \item Let  $h$ be a homomorphism. If  $L$ is regular, so is  $h(L)$
\end{itemize}
\begin{defn}[Homomorphism]
    Let $\Sigma$ and  $\Gamma$ be two alphabets, and suppose  $h$ is a mapping from  $\Sigma$ to
    $\Gamma^*$.  $h$ can be extended to strings as follows:
     \begin{itemize}
         \item $h(\epsilon)=\epsilon$
         \item  $h(aw)=h(a)\cdot h(w)$ for any  $a\in\Sigma$,  $w\in\Sigma^*$
    \end{itemize}
\end{defn}
\end{multicols}
\end{document}
