\documentclass[frenchspacing,9pt,landscape,a4paper]{article}
\usepackage{amssymb,amsmath,amsthm,amsfonts}
\usepackage{multicol,multirow}
\usepackage{calc}
\usepackage{ifthen}
\usepackage[landscape]{geometry}
\usepackage[colorlinks=true,citecolor=blue,linkcolor=blue]{hyperref}
\usepackage[linewidth=1pt]{mdframed}
\usepackage{enumitem}
\usepackage{times} % Times New Roman font to fit more content

% Even more efficient font
%\usepackage[bitstream-charter]{mathdesign}
\usepackage{charter}
\usepackage[T1]{fontenc}

\usepackage{graphicx}

\setlist[itemize]{align=parleft,left=0pt..1em}
\setlist[enumerate]{align=parleft,left=0pt..1em}

\ifthenelse{\lengthtest { \paperwidth = 11in}}
    { \geometry{top=.5in,left=.5in,right=.5in,bottom=.5in} }
	{\ifthenelse{ \lengthtest{ \paperwidth = 297mm}}
		{\geometry{top=0.8cm,left=0.8cm,right=0.8cm,bottom=0.8cm} }
		{\geometry{top=0.8cm,left=0.8cm,right=0.8cm,bottom=0.8cm} }
	}
\pagestyle{empty}
\makeatletter
\renewcommand{\section}{\@startsection{section}{1}{0mm}%
                                {-1ex plus -.5ex minus -.2ex}%
                                {0.5ex plus .2ex}%x
                                {\normalfont\large\bfseries}}
\renewcommand{\subsection}{\@startsection{subsection}{2}{0mm}%
                                {-1explus -.5ex minus -.2ex}%
                                {0.5ex plus .2ex}%
                                {\normalfont\normalsize\bfseries}}
\renewcommand{\subsubsection}{\@startsection{subsubsection}{3}{0mm}%
                                {-1ex plus -.5ex minus -.2ex}%
                                {1ex plus .2ex}%
                                {\normalfont\small\bfseries}}
\makeatother
\setcounter{secnumdepth}{0}
\setlength{\parindent}{0pt}
\setlength{\parskip}{0pt plus 0.5ex}
\setlength{\columnseprule}{0.1pt}

\newcommand{\BR}{\mathbb R}
\newcommand{\BC}{\mathbb C}
\newcommand{\BF}{\mathbb F}
\newcommand{\BQ}{\mathbb Q}
\newcommand{\BZ}{\mathbb Z}
\newcommand{\BN}{\mathbb N}

\newcommand{\mb}[1]{\mathbf #1}

\newcommand{\floor}[1]{\left\lfloor #1 \right\rfloor}
\newcommand{\ceiling}[1]{\left\lceil #1 \right\rceil}
\newcommand{\abs}[1]{\left\lvert #1 \right\rvert}
\newcommand{\norm}[1]{\left\lVert #1 \right\rVert}
\newcommand{\innerproduct}[2]{\left\langle #1, #2 \right\rangle}

\newcommand{\ddx}{\frac{d}{dx}}
\newcommand{\curl}{\text{curl }}
\newcommand{\dvg}{\text{div }}

\newcommand{\adj}{\text{adj }}

\DeclareMathOperator*{\argmax}{arg\,max}
\DeclareMathOperator*{\argmin}{arg\,min}
\DeclareMathOperator{\pr}{Pr}
\DeclareMathOperator{\var}{Var}
\DeclareMathOperator{\diam}{diam}

\theoremstyle{remark}
\newtheorem*{thm}{\textbf{Theorem}}
\newtheorem*{defn}{\textbf{Definition}}
\newtheorem*{prop}{\textbf{Proposition}}
\newtheorem*{lem}{\textbf{Lemma}}
\newtheorem*{cor}{\textbf{Corollary}}
\newtheorem*{rem}{\textbf{Remark}}
%\newtheorem*{rem}{\textbf{Comment}}
\newtheorem*{eg}{\textbf{Example}}

% -----------------------------------------------------------------------
\title{MA3209 cheatsheet}

\begin{document}

\raggedright
\footnotesize

\begin{multicols}{4} % can work with 4 too
\setlength{\premulticols}{1pt}
\setlength{\postmulticols}{1pt}
\setlength{\multicolsep}{1pt}
\setlength{\columnsep}{2pt}
\begin{mdframed}
\begin{center}
    \large{\textbf{MA3209 Metric and Topological Spaces}} \\
    \normalsize{\textbf{AY24/25 Semester 1}}\\
    \small{by Isaac Lai}
\end{center}	
\end{mdframed}
\section{Definitions}
\begin{itemize}
    \item \textbf{Topology}: collection  $\mathcal{T}$ of subsets of  $X$ such that
     \begin{enumerate}
         \item $\emptyset,X\in\mathcal{T}$
         \item  (Closure under arbitrary union) $\{U_\alpha\}_{\alpha\in I}\in\mathcal{T}\implies\bigcup_{\alpha\in
             I}U_\alpha\in\mathcal{T}$
         \item (Closure under finite intersection)  $\{U_1,\dots,U_n\}\in\mathcal{T}\implies\bigcap_{i=1}^n
             U_i\in\mathcal{T}$
     \end{enumerate} $(X,\mathcal{T})$ is a \textbf{topological space}, and  $U\subset X$ is \textbf{open}
     if  $U\in\mathcal{T}$
 \item \textbf{Basis}: collection  $\mathcal{B}$ of  $X$'s subsets such that
     \begin{enumerate}
         \item $\mathcal{B}$ covers  $X$
         \item  $\forall x\in X$ and  $B_1,B_2\in\mathcal{B}$ such that $x\in B_1\cap B_2$, $\exists
             B\in\mathcal{B}$ such that  $x\in B\subset B_1\cap B_2$
     \end{enumerate} The topology generated by $\mathcal{B}$ is  
     \[\mathcal{T}=\{U\subset X:\forall x\in U,\exists B\in\mathcal{B}\text{ such that }x\in B\subset U\}\]
 \item $\mathcal{T}'$ is \textbf{coarser} than  $\mathcal{T}$ if  $\mathcal{T}'\subset\mathcal{T}$ (reverse
     is \textbf{finer})
 \item \textbf{Subbasis} $\mathcal{S}$: collection of subsets of  $X$ whose union equals
    $X$. The \textbf{topology generated by $\mathcal{S}$} is a collection $\mathcal{T}$ of all unions of
    finite intersection of sets in  $\mathcal{S}$
%\item A \textbf{metric} on a set $X$ is a function  $d:X\times X\to\BR$ such that
%     \begin{enumerate}
%         \item (Nonnegativity) $d(x,y)\geq 0\ \forall x,y\in X$
%         \item (Positive definiteness)  $d(x,y)=0\iff x=y$
%         \item (Symmetry)  $d(x,y)=d(y,x)$
%         \item (Triangle inequality)  $d(x,y)+d(y,z)\geq d(x,z)\ \forall x,y,z\in X$
%     \end{enumerate} $(X,d)$ is a \textbf{metric space}. If only 1, 3, and 4 hold, and $d(x,x)=0$ for all
%      $x\in X$, then $d$ is a
%     \textbf{pseudo-metric}. If only 1, 2, and 4 hold,  $d$ is a \textbf{quasi-metric}.
\item A \textbf{norm} on a $\mathbb{K}$-vector space  $V$ is a function  $\norm{\cdot}:V\to\BR$ that
    satisfies
     \begin{enumerate}
         \item (Nonnegativity) $\norm{x}\geq 0\ \forall x\in V$ 
         \item (Positive definiteness) $\norm{x}=0\iff x=0$
         \item  (Absolute homogeneity) $\norm{\lambda x}=\abs{\lambda}\norm{x}$
             $\forall\lambda\in\mathbb{K}$ and $\forall x\in V$ 
         \item (Triangle inequality)  $\norm{x+y}\leq\norm{x}+\norm{y}$
    \end{enumerate}
\item Let $A,B$ be nonempty subsets of a metric space  $(X,d)$.
     \begin{itemize}
         \item \textbf{Distance}: $d(A,B)=\inf\{d(x,y):x\in A,y\in B\}$
         \item \textbf{Diameter} of $A\subset X$ is $\diam(A)=\sup\{d(x,y):x,y\in A\}$. $A\subset X$ is bounded if  $\diam(A)<+\infty$
    \end{itemize}
        \item The topology on $X$ \textbf{induced} by a metric  $d$ is the topology generated by
            $\mathcal{B}_d$.
        \item A topology  $\mathcal{T}$ on  $X$ is \textbf{metrizable} if there is a metric on  $X$ that induces
            $\mathcal{T}$
    \item Let $(Y,\mathcal{T}_Y)$ be a topological space and $X\subset Y$. Then  $\mathcal{T}_X=\{U\cap
    X:U\in\mathcal{T}_Y\}$ is the \textbf{subspace topology} on $X$.
\item Given a subset $A$ of a metric space  $(X,d)$, the \textbf{restriction} of  $d$ to  $A$ is the metric
     $d_A(x,y)=d(x,y)\ \forall x,y\in A$. The topology induced by this metric is the subspace topology.
 \item $A\subset X$ is closed if  $X\setminus A\in\mathcal{T}$
 \item Let $(X,\mathcal{T})$ be a topological space and  $A\subset X$.
     \begin{enumerate}
         \item \textbf{Interior} of $A$: $\mathring{A}=\bigcup_{U\in\mathcal{T},U\subset A}U$.
         \item \textbf{Closure} of  $A$: $\overline{A}=\bigcap_{X\setminus
             G\in\mathcal{T},G\supset A}G$.
         \item \textbf{Boundary} of  $A$: $\partial A=\overline{A}-\mathring{A}$.
    \end{enumerate}
\item Let $X$ be a topological space and  $A\subset X$. A point  $x\in X$ is a \textbf{limit point} of  $A$
    if every open  $U\subset X$ containing  $x$ intersects  $A\setminus\{x\}$.
\item A sequence $\{x_i\}_{i=1}^\infty$ in a topological space $X$ converges to
     $x\in X$ (i.e. $x_i\to x$) if for any neighbourhood  $U\ni x$,  $\exists N>0$ such that  $x_k\in U$ for all 
     $k>N$
 \item Let $X$ and  $Y$ be topological spaces. $f:X\to Y$ is \textbf{continuous} if for any open
    $U\subset Y,f^{-1}(U)\subset X$ is open.
\item Let $(X,d_X)$ and  $(Y,d_Y)$ be two metric spaces. A map  $f:X\to Y$ is \textbf{uniformly continuous}
    on  $X$ if for any  $\epsilon>0$, there exists  $\delta>0$ such that if  $x,y\in X$ satisfy
    $d_X(x,y)<\delta$, then  $d_Y(f(x),f(y))<\epsilon$.
\item Let $f_i:X\to Y$ be a sequence of maps from a set  $X$ to a metric space  $(Y,d)$:
     \begin{itemize}
         \item $\{f_i\}_{i=1}^\infty$ \textbf{converges pointwise} to  $f:X\to Y$ if  $f_i(x)\to f(x)$ for
             any  $x\in X$.
         \item  $\{f_i\}_{i=1}^\infty$ \textbf{converges uniformly} to  $f:X\to Y$ if for any $\epsilon>0$,
             there exists  $N>0$ such that for all  $i\geq N$ and any  $x\in X$,
             $d(f_i(x),f(x))<\epsilon$.
    \end{itemize}
\item For all $\alpha\in\Lambda$, the map  $\pi_{X_\alpha}:\prod_{\alpha\in\Lambda}X_\alpha\to
             X_\alpha$ defined by  $(x_\alpha)_{\alpha\in\Lambda}\mapsto x_\alpha$ is the
             \textbf{projection} to the  $\alpha$-th factor.
         \item If $(X_\alpha,\mathcal{T}_\alpha)_{\alpha\in\Lambda}$ are topological spaces, the
            \textbf{product topology} on  $\prod_{\alpha\in\Lambda}X_\alpha$ is the topology generated by
            the subbasis
            $\mathcal{S}=\{\pi_{X_\alpha}^{-1}(U_\alpha):\alpha\in\Lambda,U_\alpha\in\mathcal{T}_\alpha\}$.
        \item If  $(X_\alpha,\mathcal{T}_\alpha)_{\alpha\in\Lambda}$ are topological spaces, the
            \textbf{box topology} on  $(X_\alpha,\mathcal{T}_\alpha)_{\alpha\in\Lambda}$ is the topology
            generated by the basis  $\mathcal{B}=\{\prod_{\alpha\in\Lambda}U_\alpha:U_\alpha\subset
            X_\alpha\text{ is open}\}$.
        \item The product and box topologies are the \textbf{same for finite} product but
    \textbf{different for infinite} product.
\item If $(X_1,d_{X_1}),\dots,(X_n,d_{X_n})$ are metric spaces, there are two common metrics on
    $X_1\times\cdots\times X_n$:
    \begin{align*}
        d_1((x_1,\dots,x_n),(y_1,\dots,y_n))&=\sum_{i=1}^n d_{X_i}(x_i,y_i)\\
        d_\infty((x_1,\dots,x_n),(y_1,\dots,y_n))&=\max_{i=1,\dots,n}(d_{X_i}(x_i,y_i)) 
    \end{align*}
\item Let $X$ and  $Y$ be topological spaces.
     \begin{itemize}
         \item A surjective map $p:X\to Y$ is a \textbf{quotient map} if  $V\subset Y$ is open  $\iff
             p^{-1}(V)\subset X$ is open.
         \item A continuous map  $f:X\to Y$ is \textbf{open (closed)} if  $f(U)$ is open (closed) for any open
             (closed)  $U\subset X$.
    \end{itemize} If a surjective continuous map is open or closed $\implies$ it is a quotient map. Composition of quotient maps is also a quotient map.
\item Let $f:X\to Y$ be a surjective continuous map and  $A\subset X$. Then  $A$ is a \textbf{saturated set}
    wrt  $f$ if  $A=f^{-1}(S)$ for some  $S\subset Y$. Equivalently  $A=f^{-1}(f(A))$.
\item Let $X$ be a topological space and let  $X^*$ be the cells of a partition of  $X$. Let  $p:X\to X^*$ be
    the surjective map that sends each point in  $X$ to the subset that contains it.  $X^*$ equipped
    with the quotient topology induced by  $p$ is a \textbf{quotient space} of $X$.
    \item A topological space $X$ is  $T_1$ if for any distinct $x,y\in X$, there exists an open set  $U\subset X$ such
             that  $x\in U$ but  $y\notin U$.
         \item  $X$ is  $T_2$ or \textbf{Hausdorff} if for any distinct $x,y\in X$, there exist open
             neighbourhoods  $U,V$ of  $x,y$ respectively such that they are disjoint.
         \item  Let $X$ be a topological space. $\forall x\in X$, a \textbf{countable basis of $X$ at  $x$} is a countable collection
             $\mathcal{B}$ of open sets in  $X$ that contain  $x$ such that every open set in  $X$ that
             contains  $x$ also contains some  $B\in\mathcal{B}$.
         \item  $X$ is \textbf{first countable} if there is a countable basis of  $X$ at  $x$ for every
             $x\in X$.
         \item $X$ is \textbf{compact} if every open cover ($\{U_\alpha\}_{\alpha\in\Lambda}$ s.t. their
             union is  $X$) of  $X$ admits a finite subcover.
         \item $Y\subset X$ is a compact subspace  $\iff$ every collection  $\mathcal{U}$ of open sets in  $Y$ such
    that $Y\subset\bigcup_{U\in\mathcal{U}} U$ admits a finite sub-collection
    $\mathcal{U}'\subset\mathcal{U}$ such that  $Y\subset\bigcup_{U\in\mathcal{U}'}U$.
\item A collection $\mathcal{G}$ of subsets of  $X$ has the \textbf{finite intersection property} if every
    finite sub-collection  $\{G_1,\dots,G_n\}\subset\mathcal{G}$ satisfies $\bigcap_{i=1}^n
    G_i\neq\emptyset$.   
\item A point $x$ in a topological space is \textbf{isolated} if $\{x\}$ is open in  $X$.
\item A topological space $X$ is \textbf{limit point compact} if every infinite subset of  $X$ has a limit
    point in  $X$. Converse: No limit points $\implies$ $X$ finite 
\item Let $X$ be a topological space.  $X$ is \textbf{sequentially compact} if every sequence in  $X$ has a
    convergent subsequence.
\item Let $X$ be a metric space, and let  $\mathcal{U}$ be an open cover of  $X$. A number  $\delta>0$ is a
    \textbf{Lebesgue number} for $\mathcal{U}$ if for all subsets  $S\subset X$ such that
    $\diam(S)<\delta$, there exists  $U\in\mathcal{U}$ such that  $S\subset U$.
\item A metric space $X$ is \textbf{totally bounded} if for all $\epsilon>0$, there exists a finite cover of
     $X$ by balls of radius  $\epsilon$.
\item A sequence of points $(x_i)_{i=1}^{\infty}$ in a metric space is a \textbf{Cauchy sequence} if $\forall \epsilon > 0$, $\exists N > 0$ such that $d(x_n, x_m) < \epsilon$, $\forall m,n > N$.
\item A metric space is \textbf{complete} if every Cauchy sequence converges
\item A topological space is \textbf{locally compact at} $x\in X$ if  $\exists$ compact  $C\subset X$ and
    open  $U\subset X$ s.t.  $x\in\subset U\subset C$. If $X$ locally compact at all  $x\in X$ then it is
    locally compact
\item $f:X\to Y$ is a \textbf{homeomorphism} if  $f$ is a bijective continuous map with continuous inverse
    $f^{-1}$
\item If $Y$ is compact Hausdorff and  $\exists$ map  $h:X\to Y$ s.t.  $\overline{h(X)}=Y$ and  $h$ is a
    homeomorphism onto its image, then  $Y$ is a \textbf{compactification} of $X$. If  $Y\setminus h_Y(X)$
    is a point, then  $Y$ is a \textbf{one-point compactification} of  $X$
\item Let $(Y,d)$ be a metric space and $\rho$ the metric on $Y$ given by
\[ \rho(x,y) = \frac{d(x,y)}{1 + d(x,y)}. \]
The \emph{uniform metric} on $Y^{\Lambda} = \prod_{\alpha \in \Lambda} Y$ is the metric given by
\[ \bar{\rho}(x,y) = \sup\{\rho(\pi_{\alpha}(x), \pi_{\alpha}(y)) : \alpha \in \Lambda\}. \]
The \emph{uniform topology} on $Y^{\Lambda}$ is the topology generated by the uniform metric. It is finer
than the product topology and coarser than the box topology; these three are all different if $\Lambda$ is
infinite
\item Let $X$ be a topological space and  $(Y,d)$ a metric space.  $\mathcal{C}(X,Y)=\{f\in Y^X:f\text{ is
    continuous}\}$,  $\mathcal{B}(X,Y)=\{f\in Y^X:f(X)\subset Y\text{ has bounded diameter}\}$
\item Let $(X,d_X)$ and $(Y,d_Y)$ be two metric spaces. We say $f:(X,d_X) \to (Y,d_Y)$ is an isometric embedding if for all $a,b \in X$, $d_X(a,b) = d_Y(f(a),f(b))$. We say $f$ is an isometry if it is a surjective isometric embedding. 
\item If $(X,d_X)$ is a metric space, then a \textit{metric completion} of $X$ is a complete metric space $(Y,d_Y)$ and an isometric embedding $\phi: X \to Y$ such that $\overline{\phi(X)} = Y$.
\item A \textbf{separation} of a topological space $X$ is a pair  $U,V$ of disjoint, nonempty open subsets
    whose union is  $X$.  $X$ is \textbf{connected} if there does not exist a separation of  $X$
\item Given $x,y\in X$, a \textbf{path} from  $x$ to  $y$ is a continuous map  $f:[a,b]\to X$ s.t.
    $f(a)=x$ and  $f(b)=y$.  $X$ is \textbf{path connected} if  $\forall x,y\in X$, there is a path from
    $x$ to  $y$    
\item Let $X$ be a topological space.  $\forall x,y\in X$, define  $x\sim y$ if  $\exists$ connected subset
    $C\subset X$ s.t.  $x,y\in C$. The equivalence classes of  $\sim$ are the \textbf{connected components}
    of  $X$
\item Let $X$ be a topological space.  $\forall x,y\in X$, define  $x\overset{p}\sim y$ if $\exists$ a path
    in $X$ from  $x$ to  $y$.  $\overset{p}\sim$ are called \textbf{path components}
\item $X$ is locally (path) connected at  $x$ if  $\forall$ open  $U\subset X$ containing  $x$,  $\exists$
    (path) connected open set $V\subset X$ s.t.  $x\in V\subset U$. $X$ is locally (path) connected if it
    is locally (path connected) at every  $x\in X$   
\item A topological space $X$ is \textbf{second countable} if it has a countable basis, i.e. there exists
    some countable collection  $\mathcal{U}$ of open sets in  $X$ s.t. every open subset of  $X$ can be
    written as a union of elements in  $\mathcal{U}$. Second countable $\implies$ first countable 
\item $X$ is a \textbf{Lindelof} space if every open cover has a countable subcover
\item A $T_1$ topological space $X$ is \textbf{regular} or  $T_3$ if $\forall x\in X$ and every closed
    $B\subset X$ s.t.  $x\notin B$,  $\exists$ disjoint open sets  $U,V\subset X$ s.t.  $x\in U$ and
    $B\subset V$.
\item A $T_1$ topological space $X$ is \textbf{normal} or  $T_4$ if $\forall$ closed and disjoint
    $A,B\subset X$  $\exists$ disjoint open sets  $U,V\subset X$ s.t.  $A\subset U$ and  $B\subset V$    
\item Let $A,B\subset X$.  $A$ and  $B$ are separated by a continuous function if  $\exists$ continuous
    $f:X\to[0,1]$ s.t.  $f(A)=0$ and  $f(B)=1$.
\item  $X$ is \textbf{completely regular} if it is $T_1$ and $\forall x\in X$ and closed  $A\subset X$ s.t.
    $x\notin A$,  $\{x\}$ and  $A$ are separated by a continuous function
\item  $X$ is \textbf{completely normal} if it is $T_1$ and $\forall A,B\subset X$ closed disjoint,  $A$
    and  $B$ are separated by a continuous function
\item Let $X,Y$ be two topological spaces and  $f:X\to Y$ be an injective continuous map.  $f$ is a
    topological embedding if  $f$ is a homeomorphism between  $X$ and  $f(X)$    
\item Let $X$ be a topological space and  $(Y,d)$ a metric space. Given  $f\in Y^X$, compact  $C\subset X$,
    and  $\epsilon>0$,  $B(C,f,\epsilon)=\{g\in Y^X:\sup_{x\in C}{d(f(x),g(x))}<\epsilon\}$ is the topology
    of compact convergence.
\item Let $X,Y$ be topological spaces.  $\forall$ compact  $C\subset X$ and open $U\subset Y$,
    $S(C,U)=\{g\in\mathcal{C}(X,Y):g(C)\subset U\}$ forms a basis of the compact-open topology.
\item A topological space  $X$ is compactly generated if it satisfies either
     \begin{itemize}
         \item $A\subset X$ open  $\iff$  $A\cap C$ open in  $C$ for every compact  $C\subset X$
         \item  $B\subset X$ closed  $\iff$  $B\cap C$ closed in  $C$ for every compact  $C\subset X$
    \end{itemize}
\item Let $(Y,d)$ be a metric space and  $\mathcal{Y}\subset\mathcal{C}(X,Y)$. If  $x_0\in X$, then
    $\mathcal{Y}$ is equicontinuous at  $x_0$ if $\forall\epsilon>0$,  $\exists$ open  $U\subset X$
    containing  $x_0$ s.t. $\forall x\in U$,  $\forall f\in\mathcal{Y}$,  $d(f(x),f(x_0))<\epsilon$.
    $\mathcal{Y}$ is equicontinuous if it is equicontinuous at every  $x_0\in X$
\end{itemize}

\section{Results}
\begin{itemize}
    \item Let $\mathcal{B},\mathcal{B}'$ be bases of topologies  $\mathcal{T},\mathcal{T}'$ respectively on  $X$.
    TFAE:
     \begin{enumerate}
         \item $\mathcal{T}'$ is finer than  $\mathcal{T}$
         \item  $\forall B\in\mathcal{B}$,  $\forall x\in B$,  $\exists B'\in\mathcal{B}'$ such that  $x\in
             B'\subset B$
    \end{enumerate}
 \item $A\subset X$ open  $\iff$  $\forall a\in A$ $\exists$ open  $U_a\subset A$ such that  $a\in U_a$
     (all $a\in A$ are interior points)
 \item $A\subset X$ is closed $\iff$ to $\forall$ open  $U$
     containing  $a\in A$,  $U\cap A\neq\emptyset$ (all limit points are contained within $A$)         
 \item $\mathring{A}\subset A\subset\overline{A}$, $\mathring{A}=A\iff A$ is open, $\overline{A}=A\iff A$ is closed.
 \item Let $X$ be a topological space,  $A\subset X$.
     \begin{enumerate}
         \item $x\in\overline{A}\iff\forall$ open  $U$ containing  $x$,  $U\cap A\neq\emptyset$.
         \item If  $A'$ is the set of limit points of  $A$, then  $\overline{A}=A\cup A'$.
    \end{enumerate}
\item $f$ is continuous $\iff$ $\forall A\subset X$,  $f(\overline{A})\subset\overline{f(A)}$
             $\iff$ For any closed set  $B\subset Y$,  $f^{-1}(B)\subset X$ is closed $\iff$ For any  $x\in X$ and any open set  $V\subset Y$ containing  $f(x)$, there exists open $U\subset X$ containing  $x$ such that  $f(U)\subset V$ 
         \item (\textbf{Pasting Lemma}) Let $X=A\cup B$ where  $A,B\subset X$ are both closed (or both open). Let  $f:A\to Y$ and  $g:B\to Y$
    be continuous. If  $f(x)=g(x)$ for all  $x\in A\cap B$, then  $h:X\to Y$ defined by
    \begin{align*}
        h(x)=\begin{cases}
            f(x), & \text{ if } x\in A\\
            g(x), & \text{ if } x\in B
        \end{cases}
    \end{align*} is continuous.
    \item Let $(X,d_X)$ and  $(Y,d_Y)$ be metric spaces. $f:X\to Y$ is uniformly continuous iff for any
        two sequences  $\{x_i\}_{i=1}^\infty$ and  $\{y_i\}_{i=1}^\infty$ in  $X$ such that  $d_X(x_i,y_i)\to
        0$, $d_Y(f(x_i),f(y_i))\to 0$.
    \item Let $\{X_{\alpha}\}_{\alpha\in\Lambda}$ be topological spaces. For any  $\alpha\in\Lambda$, let
    $\pi_{X_\alpha}:\prod_{\alpha\in\Lambda}X_\alpha\to X_\alpha$ be the projection to the  $\alpha$-th
    factor:
     \begin{enumerate}
         \item The product topology on $\prod_{\alpha\in\Lambda}X_\alpha$ is the coarsest topology such
             that  $\pi_{X_\alpha}$ is continuous for any  $\alpha\in\Lambda$.
         \item Let  $Y$ be a topological space, and for any  $\alpha\in\Lambda$, let  $f_\alpha:Y\to
             X_\alpha$. The map  $f=\prod_{\alpha\in\Lambda} f_\alpha:Y\to\prod_{\alpha\in\Lambda}X_\alpha$
             defined by  $y\mapsto(f_\alpha(y))_{\alpha\in\Lambda}$ is continuous iff  $f_\alpha$ is
             continuous for every  $\alpha\in\Lambda$.
    \end{enumerate}
\item Let $X$ be a topological space and  $f,g:X\to\BR$ be continuous $\implies$ $f+g$,  $f-g$, and  $f\cdot g$
    are continuous. Also, if  $0\notin g(X)$, then  $\frac{f}{g}$ is continuous.
\item Let $(X,d)$ be a metric space. Then  $\rho:X\times X\to\BR$ given by
    $\rho(x,y)=\frac{d(x,y)}{1+d(x,y)}$ is a metric and its diameter is less than $1$. Furthermore  $\rho$
    and  $d$ induce the same topology on  $X$.
\item Let $(X_i,d_{X_i})_{i=1}^\infty$ be metric spaces for all  $i$ and let
    $\rho_{X_i}=\frac{d_{X_i}(x,y)}{1+d_{X_i}(x,y)}$ for all $x,y\in X_i$. Then  $d:\prod_{i=1}^\infty
    X_i\times\prod_{i=1}^\infty X_i\to\BR$ given by
    \[d(x,y)=\sup\{\frac{1}{i}\rho_{X_i}(x_i,y_i):i\in\BZ\}\] induces the product topology on
    $\prod_{i=1}^\infty X_i$
  \item $f$ is a quotient map  $\iff f$ sends every saturated (wrt $f$) open (closed) set to an open
             (closed) set.
         \item If $f$ is a quotient map and  $A\subset X$ is saturated and open (closed), then
             $f|_A:A\to f(A)$ is also a quotient map.
         \item $X$ is  $T_1\iff\forall x\in X$, $\{x\}$ is closed. Hence finite sets in metric spaces are
    closed.
         \item Let $X$ be a topological space, and $A\subset X$. If there exists a sequence  $(x_i)_{i=1}^\infty\subset A$ such that
             $x_i\to x$ as  $i\to\infty$, then  $x\in\overline{A}$. The converse is true if  $X$ is first
             countable.
         \item Let  $f:X\to Y$. If  $f$ is continuous, then for any sequence  $(x_i)_{i=1}^\infty\subset X$
             such that  $x_i\to x$ as  $n\to\infty$, we have  $f(x_i)\to f(x)$ as  $i\to\infty$. The
             converse holds if  $X$ is first countable.
         \item Every closed subspace of a compact space is compact.
         \item (\textbf{Tube Lemma}) Let $X$ be a topological space and  $Y$ be a compact topological space. If  $N\subset X\times Y$ is an
    open set that contains  $\{(x_0,y):y\in Y\}$, then $N$ contains $W\times Y$ for some  $W\subset X$ that
    contains $x_0$.
\item If $X$ and  $Y$ are compact topological spaces, then  $X\times Y$ is compact.
\item $X$ compact $\iff$ $X$ has the following property: Let
    $\mathcal{G}$ be a collection of closed sets in  $X$. If  $\mathcal{G}$ has the finite intersection
    property, then  $\bigcap_{G\in\mathcal{G}} G\neq\emptyset$. Idea: consider $\mathcal{U}=\{X\setminus
    G:G\in\mathcal{G}\}$, use FIP, note  $X\setminus(\bigcap_{G\in\mathcal{G}})=\emptyset$
\item If $X$ is compact and  $\{G_i\}_{i=1}^\infty$ is a nested (i.e. $G_{i+1}\subset G_i$ for all
    $i\in\BZ^+$) sequence of closed subsets in $X$, then  $\bigcap_{i=1}^\infty G_i\neq\emptyset$.
\item Let $X$ be a non-empty, compact, Hausdorff space. If  $X$ has no isolated points, then  $X$ is
    uncountable.
\item Compact $\implies$ limit point compact
\item If $X$ is (any type of) compact metric space, then every open cover of $X$ has a Lebesgue number. 
\item (Any type of) compactness and metrizable $\implies$ totally bounded. Idea: contradiction, construct
    sequence  $x_n\in X\setminus(\bigcup_{i=1}^{n-1}B_{\epsilon_0}(x_i))$ which has no convergent
    subsequence
\item If $X$ is metrizable, compact $\iff$ limit point compact $\iff$ sequentially compact.
\item Let $f:(X,d_X)\to(Y,d_Y)$ be continuous. If  $X$ is compact, then  $f$ is uniformly continuous.
\item Totally bounded $\implies$ finite diameter
\item $X$ locally compact and Hausdorff iff  $\exists$ compact Hausdorff space  $Y$ and map  $h_Y:X\to Y$
    s.t. (1)  $h_Y$ is a homeomorphism onto its range and (2)  $Y\setminus h_Y(X)$ is a single point.
    Furthermore, if $(Y,h_Y)$ and  $(Y',h_{Y'})$ are two such spaces and maps,  $\exists$ homeomorphism
    $f:Y\to Y'$ s.t.  $f|_{h_Y(X)} = h_{Y'} \circ h_Y^{-1}|_{h_Y(X)} : h_Y(X) \to h_{Y'}(X)$
\item $X$ is Hausdorff, non-compact, locally compact $\implies$ it admits a unique one-point
    compactification
\item Let $X$ be a Hausdorff topological space. Then $X$ is locally compact is equivalent to for any $x \in X$, for any open $U \subset X$ such that $x \in U$, there exists open $V \subset X$ such that $x \in V$, $\bar{V} \subset U$ and $\bar{V}$ is compact.
\item Let $X$ be a locally compact space. $A\subset X$ is closed or  $X$ is Hausdorff and  $A$ is open
    $\implies$  $A$ locally compact
\item  $X$ is a homeomorphism to an open subset of a compact Hausdorff space  $\iff$  $X$ is locally
    compact and Hausdorff
\item $(Y,d)$ complete  $\implies$  $(Y^\Lambda,\overline{\rho})$ complete   
\item $\mathcal{C}(X,Y),\mathcal{B}(X,Y)\subset Y^X$ are closed in the uniform topology. In particular,
    $(Y,d)$ complete  $\implies$  $(\mathcal{C}(X,Y),\overline{\rho})$ and
    $(\mathcal{B}(X,Y),\overline{\rho})$ complete
\item Let $(X,d)$ be a metric space. Then there is an isometric embedding $\phi$ of $X$ into a complete metric space $Y$ such that $\phi(X) \subset Y$ is dense. Furthermore, if $(Y',d_{Y'})$ is a complete metric space and $\phi': X \to Y'$ is an isometric embedding such that $\overline{\phi'(X)} = Y'$, then there exists an isometry $f: Y \to Y'$ such that
\[
f|_{\phi(X)} = \phi' \circ \phi^{-1}: \phi(X) \to \phi'(X).
\]
\item $X$ is connected  $\iff$ the only sets in  $X$ that are open and closed are  $\emptyset$ and  $X$
\item $\{A_\alpha\}_{\alpha\in\Lambda}$ is a collection of connected subsets of  $X$ s.t.
    $\bigcap_{\alpha\in\Lambda}A_\alpha\neq\emptyset$  $\implies$  $\bigcup_{\alpha\in\Lambda}A_\alpha\subset X$ is
    connected
\item $A\subset X$ connected and  $A\subset B\subset\overline{A}$  $\implies$  $B$ connected
\item $f:X\to Y$ continuous and  $A\subset X$ connected  $\implies$  $f(A)\subset Y$ connected
\item $X,Y$ connected  $\implies$  $X\times Y$ connected    
\item Every connected component of $X$ is connected
\item $X$ is locally (path) connected  $\iff$  $\forall$ open  $U\subset X$, each (path) connected
    component of  $U$ is open in  $X$    
\item If $X$ is locally path connected, then connected components and path components are the same    
\item Suppose $X$ is second countable. Then (1)  $X$ is Lindelof and (2) there exists a countable subset
    $A\subset X$ that is dense, i.e.  $\overline{A}=X$. The converse of both of them are true if  $X$ is
    metrizable   
\item $X$ is regular  $\iff$  $\forall x\in X$,  $\forall U\subset X$ containing  $x$,  $\exists$ open
    $V\subset X$ containing  $x$ s.t.  $\overline{V}\subset U$   
\item Every metrizable space is normal
\item $X$ is a regular topological space with a countable basis $\implies$  $X$ is normal    
\item (Urysohn's metrization theorem) $X$ is regular with countable basis  $\implies$ $X$ is metrizable    
\item (Urysohn's lemma) $X$ is normal  $\implies$  $X$ is completely normal
\item (Tychonoff's theorem) The product of compact spaces is compact, i.e. if
    $\{X_\alpha\}_{\alpha\in\Lambda}$ is a family of compact spaces, then
    $X=\prod_{\alpha\in\Lambda}X_\alpha$ is compact wrt product topology
\item (Arzela-Ascoli theorem) Let $X$ be a topological space and $(Y,d)$ a metric space. Equip
    $\mathcal{C}(X,Y)$ with the compact open topology and let  $\mathcal{Y}\subset\mathcal{C}(X,Y)$.
    \begin{enumerate}
        \item If $\mathcal{Y}$ is equicontinuous under  $d$ and  $\mathcal{Y}_a=\{f(a):f\in\mathcal{Y}\}$
            has compact closure for each  $a\in X$, then  $\overline{\mathcal{Y}}\subset\mathcal{C}(X,Y)$
            is compact
        \item Converse holds if  $X$ is locally compact and Hausdorff
    \end{enumerate}
\end{itemize}

\section{Examples}
%\subsection{Topologies}
\begin{itemize}
    \item $\mathcal{T}=\{\emptyset,X\}$ is the \textbf{trivial topology}
    \item  $\mathcal{T}=\{\text{subsets of }X\}$ is the \textbf{discrete topology}. Non-compact, has
        Lebesgue number (midterm question)
    \item $\mathcal{T}=\{X-U:U\subset X\text{ is finite}\}\cup\{\emptyset\}$ is the \textbf{cofinite
        topology}
    \item $X=\{a,b,c\}$, possible topologies include  $\{\{a\},\{a,b\},\emptyset,X\}$,
        $\{\{b,c\},\emptyset,X\}$ (and more)
    \item $X=\BR$,  $\mathcal{T}=\{(-\alpha,\alpha):\alpha\in\BR^+\}\cup\{\emptyset,\BR\}$
    \item (\textbf{HW}) Collection of unions of arithmetic sequences
    \item (\textbf{HW}) \textbf{Co-countable topology}: $U$ is open if  $U=\emptyset$ or  $X\setminus U$ is
        countable. It is not comparable with the standard topology, but finer than the co-finite topology
%\end{itemize}    
%\subsection{Metrics and norms}
%\begin{itemize}
    \item The \textbf{discrete metric} is
        \begin{align*}
            d(x,y)=
            \begin{cases}
                1, & \text{ if }x\neq y\\
                0, & \text{ if } x=y
            \end{cases}
        \end{align*} 
    \item (\textbf{HW}) Let $X$ the space of all closed subsets of  $\BR^n$. Let
        $B_\epsilon(A)=\bigcup_{a\in A}B_\epsilon(a)$ be an  $\epsilon$-neighbourhood of  $A$. Then the
        \textbf{Hausdorff metric}  $d_H(A,B)=\inf\{\epsilon>0:A\subset B_\epsilon(B)\text{ and }B\subset
        B_\epsilon(A)\}$ is a metric on  $X$. This is not a metric on the space of all subsets of $\BR^n$,
        e.g.  $A=[0,1]^n$,  $B=(0,1)^n$
    \item The \textbf{$l_p$-norm} is $V=\mathbb{K}^n,p\geq 1,\norm{x}_p=(\abs{x_1}^p+\cdots+\abs{x_n}^p)^{1
        /p},x\in\mathbb{K}^n$.
    \item The \textbf{$l_\infty$-norm} is
        $V=\mathbb{K}^n,\norm{x}_\infty=\max\{\abs{x_1},\dots,\abs{x_n}\},x\in\mathbb{K}^n$
    \item $[a,b]\subset\BR$ is closed wrt standard topology on  $\BR$
    \item Let $X=[0,1]\cup(2,3)\subset\BR$.  $[0,1]$ is both open and closed in  $X$ wrt subspace topology
        on  $X$
    \item $\{0\}\cup(1,2)\subset\BR$ has $[1,2]$ as its set of limit points wrt standard topology on  $\BR$
         \begin{itemize}
             \item $0$ is not a limit point of  $A$, e.g. $(-1 /2,1/ 2)$ open but doesn't intersect
                 $A\setminus\{0\}$
             \item  $x\in\BR\setminus(\{0\}\cup[1,2])$ is not a limit pt of  $A$
             \item Every  $x\in[1,2]$ is a limit pt of  $A$
        \end{itemize}
    \item $x$ being a lim pt of  $\{x_i\}$  $\nRightarrow$ $x_i\to x$. E.g.  $\{(-1)^n+\frac{1}{n}\}$
        doesn't converge but has lim pts $\{-1,1\}$
    \item  $x_i\to x$  $\nRightarrow$  $x$ is a lim pt of  $\{x_i\}$. E.g.  $(1,1,\dots)$ converges to  $1$
        but  $\{1\}$ has no lim pt
    \item Given a topology $\mathcal{T}_Y$ on  $Y$ and a map  $f:X\to Y$, the \textbf{pull back} topology on  $X$
        is $\mathcal{T}_X=\{f^{-1}(U):U\in\mathcal{T}_Y\}$. This is the coarsest topology on  $X$
        such that  $f$ is continuous.
    \item If $(X,d)$ is a metric space with  $A\subset X$ nonempty, then  $f:X\to\BR$ defined by  $x\mapsto
        d(x,A)$ is uniformly continuous. Idea:  $d(x,A)\leq d(x,y)+d(y,A)$, let  $\epsilon=\delta$
    \item (\textbf{Midterm}) Let $\BR^\BN$ be equipped with the box topology, and consider  $f:\BR\to\BR^\BN$ as
        $x\mapsto(x,x,\dots)$. All component functions are the identity and hence continuous, but  $f$ is
        not continuous. Idea: Let  $U=\prod_{n=1}^\infty(-\frac{1}{n},\frac{1}{n})$. If $f$ continuous,
        there should exist  $\epsilon>0$ s.t.  $(-\epsilon,\epsilon)\subset f^{-1}(U)$, but
        $\frac{\epsilon}{2}>\frac{1}{n}$
    \item Let $p : [0,1] \cup [2,3] \to [0,2]$ be a map defined by
\[ x \mapsto \begin{cases}
x & \text{if } x \in [0,1], \\
x-1 & \text{if } x \in [2,3].
\end{cases} \]
$p$ is closed but not open. By the pasting lemma, we know that $p$ is continuous. Let $A \subset [0,1] \cup [2,3]$ be closed, then
\[ A_1 = A \cap [0,1] \subset [0,1], \quad A_2 = A \cap [2,3] \subset [2,3] \]
are closed sets. This together with the definition of $p$, $[0,1]$ and $[1,2]$ are closed in $[0,2]$ shows that $p(A_1)$ and $p(A_2)$ are closed subsets of $[0,2]$ with respect to its topology. As a result we show that $p(A)$ is closed and thus $p$ is a closed map. But $p$ is not open. Let $B = (0,1)$, then $B$ is open wrt. the subspace topology in $[0,1] \cup [2,3]$. However $p((0,1)) = (0,1) \subset [0,2]$ is not open.
\item (\textbf{Midterm}) $p:(0,1)\cup(2,3)\to(0,2)$ is open but not closed by a similar argument
\item Let $X=\BR^2\setminus\{(x,y):0\leq x<1,0<y<1\}$ and $f:X\to\BR$ be defined as  $f(x,y)=x$. Then  $f$
    is surjective, continuous, not open, not closed, but a quotient map.
\item If $X$ is a topological space,  $A\subset X$ and  $p:X\to A$ is surjective, then  $\exists!$ topology
    on  $A$ (called the \textbf{quotient topology}) such that  $p$ is a quotient map where
    $\mathcal{T}=\{U\subset A:p^{-1}(U)\subset X\text{ is open}\}$
\item Let $p : \mathbb{R} \to \{a, b, c\}$ be a map defined as
\[ x \mapsto \begin{cases}
a, & \text{if } x > 0, \\
b, & \text{if } x = 0, \\
c, & \text{if } x < 0.
\end{cases} \]
Then the quotient topology on $\{a, b, c\}$ is
\[ \mathcal{T} = \{\{a\}, \{c\}, \{a,c\}, \{a,b,c\}, \emptyset\}. \]
\item Partition $X = \mathbb{R} = \mathbb{R}^- \cup \{0\} \cup \mathbb{R}^+$. Then the quotient space $X^* = \{\mathbb{R}^-, \{0\}, \mathbb{R}^+\}$.
\item Let $X = \{(x,y) : x^2 + y^2 \leq 1\}$ and decompose it as the union of
    $\bigcup_{(x,y):x^2+y^2<1}\{(x,y)\}$ and $\{(x,y) : x^2 + y^2 = 1\}$. Then $X^*$ is
\[\{\{(x,y)\} : x^2 + y^2 < 1\} \cup \{\{(x,y) : x^2 + y^2 = 1\}\}. \]
\item Let $X = \mathbb{R}$ and $p$ defined as that $p$ sends $x$ to $x + n$ for some $n \in \mathbb{Z}$
    such that $x + n \in [0,1)$. It is clear that such $n$ is unique for a fixed $x \in \mathbb{R}$.

In this setting, $X^* = [0,1)$. We may also identify $X^*$ as $S^1$ (unit circle) or $\mathbb{R}/\mathbb{Z}$.
    \item Any Hausdorff space is $T_1$
    \item Any metric space is Hausdorff
    \item If $\abs{X}\geq 2$, then the trivial topology is not  $T_1$
    \item The discrete topology is Hausdorff
    \item The cofinite topology is $T_1$. The cofinite topology is Hausdorff iff $X$ is finite. Idea: for
        $T_2$, break into finite/infinite cases, for infinite case let $U,V$ be $X$ minus finite number of
        elements, these cannot be disjoint
    \item If $X$ is infinite, then the cofinite topology on  $X$ is not metrizable
    \item Metric spaces are first countable: $\forall x\in X$,  $\{B_{1 /i}(x):i\in\BZ^+\}$ is a countable
        basis of  $X$ at  $x$
    \item The cofinite topology on an uncountable set  $X$, e.g.  $\BR$ is not first countable. Idea:
        suppose countable basis exists,  $B_i=X\setminus F_i$ for some finite  $F_i\subset X$, consider
        $y\in X\setminus(\{x\}\cup\bigcup_{i=1}^\infty F_i)$, show that  $B_i$ not subset of  $U$
    \item $X = \{\frac{1}{n} : n \in \mathbb{Z}^+\} \subset \mathbb{R}$ is not compact since
        $\{\{\frac{1}{n}\} : n \in \mathbb{Z}^+\} \subset \mathbb{R}$ is an open cover of $X$ wrt subspace topology but does not have finite subcover.
    \item $X = \{\frac{1}{n} : n \in \mathbb{Z}^+\} \cup \{0\} \subset \mathbb{R}$ is compact. Let $\mathcal{U}$ be any open cover, then $\exists U \in \mathcal{U}$ that contains $0$ and $\exists N > 0$ such that $\frac{1}{n} \in U$ for $n \geq N$. For each $n < N$, let $U_n \in \mathcal{U}$ such that $\frac{1}{n} \in U_n$. All together we obtain a finite subcover of $\mathcal{U}$ as:
$\{U_1, \ldots, U_{N-1}, U\} \subset \mathcal{U}$
\item Any metric space $X$ of infinite diameter is not compact. Let $x \in X$. Then $\{B_n(x) : n \in
    \mathbb{Z}^+\}$ is an open cover of $X$ which does not have a finite subcover.`
\item $S=\{(x,y)\in\BR^2:\abs{x}\leq \frac{1}{y^2+1}\}\subset\BR^2$ contains $\{0\}\times\BR$ but not a
    tube
\item Any unbounded metric space is not limit point compact. Idea: Pick $x_1\in X$ and $x_i\in
    B_i(x_1)\setminus B_{i-1}(x_1)$, $\{x_1,x_2,\dots\}$ has no limit points
\item Let $Y$ (where $\abs{Y}\geq 2$) be equipped with the trivial topology and $\mathbb{Z}^+$ be equipped with
    discrete topology. Let $X = \mathbb{Z}^+ \times Y$, then the product topology on $X$ is $\{A \times Y :
    A \subset \mathbb{Z}^+\}$. Every non-empty subset of $\mathbb{Z}^+ \times Y$ has a limit point, so $X$
    is limit point compact. However, $\{\{a\} \times Y : a \in A\}$ is a cover of $X$ with no finite
    subcover, so $X$ is not compact
\item $\BR^n$ with  $\ell_p$ metric has infinite diamter, so it is not totally bounded
\item $\mathbb{R}^n$ with respect to $l_p$ metric for $p \in [1, \infty]$ is complete.    
\item If $(x_i)_{i=1}^\infty$ is a Cauchy sequence in $\mathbb{R}^n$, then $\exists M > 0$ such that $x_i \in B'_M(0) = \{x \in \mathbb{R}^n : d(x, 0) \leq M\}$ for every $i$. Since $B'_M(0)$ is compact and thus sequentially compact, $(x_i)_{i=1}^\infty$ has a convergent subsequence, so $(x_i)_{i=1}^\infty$ converges.
    \item Equipped with the standard metric on $\mathbb{R}$ restricted to $\mathbb{Q}$, $\mathbb{Q}$ is not complete. Since $\mathbb{Q} \subset \mathbb{R}$ is dense, there are sequences in $\mathbb{Q}$ that converge in $\mathbb{R}$ to an irrational number. Such sequences are Cauchy but do not have a convergent subsequence in $\mathbb{Q}$.
    \item Let $d$ be the standard metric on $\mathbb{R}$, $\rho$ the metric on $\mathbb{R}$ given by
    \[
    \rho(x, y) = \frac{d(x, y)}{1 + d(x, y)},
    \]
    $D$ the metric on $\prod_{\mathbb{Z}} \mathbb{R} = \mathbb{R}^\omega$ given by
    \[
    D(x, y) = \sup \left\{ \frac{\rho(\pi_k(x), \pi_k(y))}{k} : k \in \mathbb{Z}^+ \right\},
    \]
    where $\pi_k : \mathbb{R}^\omega \to \mathbb{R}$ is the projection to the $k$-th factor.
\item (\textbf{HW}) $(\BR^w,D)$ is  complete
\item $\BR^n$ is locally compact. Let $U=B_\epsilon(x)$ and  $C=\overline{U}$
\item  $\BQ\subset\BR$ is not locally compact
\item $\BR^w$ with product topology is not locally compact
\item Let $\mathbb{D}=\{(x,y):x^2+y^2<1\}$. Then  $\overline{\mathbb{D}}$ and  $\mathbb{S}^2$ are
    compactifications of  $\mathbb{D}$, and  $\mathbb{S}^2$ is a one-point compactification
\item $U_1=\prod_{n\in\BN}\{x=(x_n)_{n\in\BN}:\abs{x_n}<2^{-n}\forall n\in\BN\}$ is open in the box
    topology but not the uniform topology on $\BR^\BN$ (does not contain uniform ball)
\item $U_2=\{x\in\BR^\BN:\rho(x,0)<0.01\}$ is open in the uniform topology but not in the product topology
    (does not contain any set of the form $\{x=(x_n)_{n\in\BN}\in\BR^\BN:x_{n_1}=\cdots=x_{n_k}=0$)
    \item The supremum metric $d_{\sup}$ on  $\mathcal{B}(X,Y)$ is
        $d_{\sup} (f,g)={\sup}\{d(f(x),g(x)):x\in X\}$. This is well defined since $\diam(f(X)\cup g(X))$ is
        bounded
    \item The trivial topology is connected. 
    \item $[-1,0)\cup(0,1]\subset X$ is not connected wrt standard topology
    \item  $\BQ\subset\BR$ is not connected wrt standard topology
        $(a,b)\subset\BR$ is connected, as are  $(a,b]$,  $[a,b)$,  $[a,b]$
    \item (Topologist's sine curve) Let $S$ be defined as $S = \{(x,y) \in \mathbb{R}^2 : y =
        \sin(\frac{2\pi}{x}), 0 < x \leq 1\}$.
Let $f: (0,1] \to S$ be defined as $t \mapsto (t, \sin\frac{2\pi}{t})$. This gives that $S = f((0,1])$ is
path connected and thus connected. In particular, $\bar{S} = S \cup (\{0\} \times [-1,1])$ is connected.
Idea: if a separation exists, then $A$ contains all  $(t,\sin 2\pi /t)$ so $B$ contains only  $(0,0)$. But
any open set around  $(0,0)$ intersects  $A$. However, TSC is not path connected (it can never leave the
y-axis)
\item $(0,1)\subset\BR$ is connected and locally connected
\item $(0,1)\cup(1,2)$ is not connected but locally connected
\item TSC is connected but not locally connected
\item $\BQ\subset\BR$ is neither connected nor locally connected    
\item $\BR^n$ is second countable since  $\{B_r(x):r\in\BQ,x\in\BQ^n\}$ is a countable basis
\item $\BR^w$ with product topology is second countable since
    $\{\prod_{n\in\Lambda}(a_n,b_n)\times\prod_{n\in\BZ\setminus\Lambda}\BR:\Lambda\text{ is
    finite},a_n<b_n,a_n,b_n\in\BQ\ \forall n\in\Lambda\}$    
\end{itemize}

\section{Exercises}
\begin{itemize}
    \item $\mathcal{T}$ is equal to the collection of all unions of elements in  $\mathcal{B}$
    \item Open balls of radius  $r$ are a basis on  $\BR^n$, this is the \textbf{standard topology}. Proof
        idea: use definition of basis, covering is obvious, for intersection use two balls and triangle
        inequality
    \item Discrete metric generates the discrete topology. Proof idea: If $r<1$,  $B_r(x)=\{x\}$, unions of
        the singletons produce all subsets of  $X$.
    \item Every  $\ell_p$ metric on  $\BR^n$ generates the standard topology. Forward: consider
        $d_p(x,y)<r$, square both sides, open brackets using inequality. Reverse: Let $\epsilon>0$,
        $\delta=\frac{\epsilon}{n^{1 /p}}$, bound by $n\cdot\max\{\abs{x_i-y_i}^2\}^{p /2}$, play games
    \item $\mathcal{B}$ is a basis for $\mathcal{T}_Y$  $\implies$  $\{B\cap X:B\in\mathcal{B}\}$ is a
        basis for  $\mathcal{T}_X$
    \item If $X\subset Y$ is open and  $U\subset X$ is open, then  $U\subset Y$ is open.
    \item (\textbf{HW}) Topology induced by subspace metric is the subspace topology
    \item (\textbf{HW}) $\triangle=\{(x,x):x\in\BR\}\subset\BR^2$ is closed wrt standard topology on
        $\BR^2$. Idea: let  $r_p$ be the distance from  $p\in\BR^2\setminus\triangle$ to  $\triangle$. Show
        that  $\BR^2\setminus=\bigcup_{p\in\BR^2\setminus\triangle}B_{r_p}(p)$
    \item (\textbf{HW}) Let $X$ be a topological space
        \begin{itemize}
            \item If $\{G_\alpha\}_{\alpha \in I}$ is an arbitrary collection of closed sets in $X$, then
                $\bigcap_{\alpha \in I} G_\alpha \subset X$ is closed.
            \item If $G_1, \dots, G_n$ are closed sets in $X$, then $\bigcup_{i=1}^n G_i \subset X$ is closed.
            \item If $Y \subset X$, then $A \subset Y$ is closed is equivalent to $A = G \cap Y$ for some closed $G
                \subset X$. Idea: consider subspace topology, then use $A=Y\setminus(Y\setminus A)=Y\setminus(H\cap
                Y)=Y\setminus H=(X\setminus H)\cap Y$
            \item If $Y \subset X$ is closed and $A \subset Y$ and $A \subset Y$ is closed, then $A \subset X$ is closed.
        \end{itemize}
    \item (\textbf{HW}) Find interior/closure/boundary of $\{(x,y)\in\BR^2:0<x\leq 1,0<y\leq 1\}$. Idea:
        Let  $\epsilon=\min\{x,y,1-x,1-y\}$ for interior (and similar for closure). Play games
    \item If $X$ equipped with discrete topology, all subsets of  $X$ have no limit pts. Idea: singletons
        $\{x\}$ are open but do not intersect any $A\setminus\{x\}$
    \item If $(X,d)$ is a metric space,  $\{x_i\}$ converging to  $x$  $\iff$  $\forall\epsilon>0,\exists
        N>0$ s.t.  $d(x_i,x)<\epsilon\forall i>N$. Idea: apply definition of convergence, note that balls
        form a basis
    \item If $\mathcal{S}$ is a subbasis for  $\mathcal{T}_Y$, then  $f:X\to Y$ continuous  $\iff$
        $f^{-1}(S)\subset X$ is open for any  $S\in\mathcal{S}$. Idea: just apply definition of subbasis
    \item Let  $X,Y,Z$ be topological spaces. Constant map, composition, inclusion map, restriction map are
        all continuous.
    \item Let $(X, d_X)$ and $(Y, d_Y)$ be two metric spaces, then $f : X \to Y$ is continuous wrt the
        topologies induced by these metrics $\iff$ $\forall x \in X$, $\forall\epsilon > 0$, $\exists\delta
        > 0$ s.t. if $y \in X$ satisfies $d_X(x, y) < \delta$, then $d_Y(f(x), f(y)) < \epsilon$. Idea:
        observe that any open set is just a union of open balls, consider $\delta$ ball around  $x\in
        X$ and $x\in f^{-1}(V)$ and  $\epsilon$ ball around  $f(x)$
    \item (\textbf{Midterm}) Let $f_i:X\to Y$ be a sequence of continuous functions from topological space  $X$ to metric
        space  $(Y,d)$.  $\{f_i\}$ converges uniformly to  $f:X\to Y$  $\implies$  $f$ is continuous. Idea:
        $f^{-1}(U)=\bigcup_{i\geq N}f_i^{-1}(B_{\epsilon /2}(f(x)))$
    \item (\textbf{Tut})  $A\subset X$, $B\subset Y$  $\implies$ product topology on  $A\times B$ induced
        by subspace topologies on  $A\subset X$, $B\subset Y$ same as subspace topology
        $A\times B\subset X\times Y$ induced by product topology on  $X\times Y$
    \item Let $n\in\BZ^+$ s.t.  $n=m_1+\cdots+m_k$ where all $m_i\in\BZ^+$. Then the product topology of
        standard topologies on $\BR^{m_1}\times\cdots\times\BR^{m_k}=\BR^n$ is the standard topology on
        $\BR^n$. Idea: forward direction, sum up all $\epsilon$ of each $B_{m_i}$ to make a  $B_\epsilon$
        in  $\BR^n$
    \item If $\mathcal{B}_1,\dots,\mathcal{B}_n$ are bases for the topological spaces
        $(X_1,\mathcal{T}_1),\dots,(X_n,\mathcal{T}_n)$ respectively, then
        $\mathcal{B}_1\times\cdots\times\mathcal{B}_n$ is a basis for  $X_1\times\cdots\times X_n$
        that generates the product topology.
    \item If $(X_1,d_{X_1}),\dots,(X_n,d_{X_n})$ are metric spaces that induce topologies
        $\mathcal{T}_1,\dots,\mathcal{T}_n$ on  $X_1,\dots,X_n$ respectively, then the metrics $d_1
        $and $d_\infty$ on  $X_1\times\cdots\times X_n$ both induce the product topology.
    \item In the infinite product case, let $(X_i,d_{X_i})_{i=1}^\infty$ be metric spaces. Given the
        metric  $d_\infty$ above, we define  $d_\infty:\prod_{i=1}^\infty X_i\times\prod_{i=1}^\infty
        X_i\to\BR$ by  $d_\infty(x,y)=\sup\{d_{X_i}(x_i,y_i):i\in\BZ^+\}$. But this is not well-defined
        as  $d_{X_i}(x_i,y_i)$ might be unbounded as  $i\to\infty$.
    \item (\textbf{Midterm}) Every compact subspace of a Hausdorff space is closed. Idea: show that no
        limit point can be found in the complement of the subspace
    \item (\textbf{HW}) If $X$ is equipped with the cofinite topology, then every subset is compact but
        only the finite sets are closed. Idea: take one set in the open cover, only finite number of points
        not covered, so patch with finite number of open sets
    \item Sequentially compact $\implies$ limit point compact. Idea: $A\subset X$, convergent subsequence
        in $A$ converges to  $a\in X$, we have points in any neighbourhood of $a$, so this must be a limit
        point. (\textbf{HW}) Converse is not true, e.g. $\{(a,\infty):a\in\BR\}$
    \item (\textbf{HW}) Open cover of $\BR$ with no Lebesgue number:  $(n-\frac{1}{\abs{n}+1},n+\frac{1}{\abs{n}+1})$
    \item If $(X,d)$ is a metric space and $\rho(x,y) = \frac{d(x,y)}{1+d(x,y)}$, then $(X,\rho)$ is totally bounded $\leftrightarrow$ $(X,d)$ is totally bounded.
    \item If $(X,d)$ and $(X',d')$ are bi-Lipschitz (i.e. $\exists A > 1$ such that $A^{-1}d(x,y) \leq
        d'(x,y) \leq Ad(x,y)$ for $\forall x,y \in X$), then $(X,d)$ is totally bounded $\iff$ $(X',d')$ is totally bounded
    \item Any subset in $(\mathbb{R}^n, l_p)$ is totally bounded $\leftrightarrow$ it is bounded.
    \item (\textbf{HW}) A Cauchy sequence converges iff it has a convergent subsequence.
    \item (\textbf{HW}) If $d$ and $d'$ are metrics on $X$ that are bi-Lipschitz, then a sequence is Cauchy in $(X,d)$ if and only if it is Cauchy in $(X',d')$.
\item For a metric space, compact  $\iff$ complete and totally bounded. Idea: consider lim pt $x\in X$ and Cauchy sequence
    where $x_i\in B_{\frac{1}{i}}(x)$ Corollary:  $G\subset\BR^n$ compact
     $\iff$ $G$ closed and bounded (Heine-Borel Thm)
 \item (\textbf{HW}) If $U,V\subset X$ is a separation of  $X$ and  $Y\subset X$ is a connected subspace,
     then  $Y\subset U$ or  $Y\subset V$
 \item (\textbf{HW}) Path connected $\implies$ connected. Idea: continuity of path implies $[a,b]$
     disconnected
 \item (\textbf{HW}) All path components are path connected and thus every path component lies in a
     connected component. Idea: path connected $\implies$ connected
 \item (\textbf{HW}) If $X$ is locally path connected and $\tilde{X}=\{\text{connected components of }X\}$,
     then the quotient topology on $\tilde{X}$ is discrete. Idea: show $U$ in quotient topology  $\iff$
     $U$ is a CC of  $X$
 \item (\textbf{HW}) First and second countability are preserved by taking products and subspaces     
 \item (\textbf{HW}) $T_4\implies T_3\implies T_2\implies T_1$
 \item (\textbf{HW}) If $X$ is copmact,  $T_2\iff T_3\iff T_4$
 \item  Let $X$ be a topological space.  $X$ is normal  $\iff$  $\forall$ closed  $A\subset X$,  $\forall$
     open $U\supset A$,  $\exists$ open  $V\supset A$ s.t.  $\overline{V}\subset U$. Idea: forward:
     consider $A$ and  $X\setminus U$ which are closed and disjoint. Reverse:  $X\setminus B$ open and
     $A\subset X\setminus B$, same for  $X\setminus A$
 \item Completely regular $\implies$ regular. Completely normal  $\implies$ normal
\end{itemize}

%\includegraphics[scale=0.21]{daren.png}
\section{HW}
\begin{itemize}
    \item $S\subset X$ is dense in  $X$ means that  $\forall x\in X$ and every neighbourhood
        $U$ of  $x$,  $U\cap S\neq\emptyset$. This is equivalent to the interior of $X\setminus S$ being
        empty and $\overline{S}=X$
    \item The limit points of $\{\frac{1}{m}+\frac{1}{n}:m,n\in\BZ^+\}\subset\BR$ are
        $\{0\}\cup\{\frac{1}{k}\mid k\in\BN\}$. The limit point of $\{\frac{\sin
        n}{n}:n\in\BZ^+\}\subset\BR$ is $0$
    \item Topological space with compact subset but closure of subset is not compact:
        $\mathcal{T}$ is standard topology on  $\BR, $$X=\{U\cup\{0\}\mid
        U\in\mathcal{T}\}\cup\{\emptyset\}$, $\{0\}$ is compact but $\overline{\{0\}}=\BR$ is not
    \item A sequence $\{x_n\}$ converges to a point  $x$ iff every subsequence has in turn a subsequence
        converging to  $x$. Idea: forward: if no subsequence exists, then there are an infinite number of
        elements lying outside a neighbourhood of  $x$.
    \item Product of two Hausdorff spaces is Hausdorff
    \item Subspace of Hausdorff space is Hausdorff
    \item $(-\sqrt{2},\sqrt{2})\cap\BQ$ is a closed and bounded subset of  $\BQ$ that is not compact. Idea:
        to show not compact, consider the cover $A_n=(-\sqrt{2}+\frac{1}{n},\sqrt{2})\cap\BQ$
    %\item $[0,1)$ and  $[0,1]$ are not homeomorphic
    %\item  $\BQ$ is not locally path connected wrt subspace topology induced from standard topology on
    %    $\BR$
\end{itemize}

\end{multicols}
\end{document}
