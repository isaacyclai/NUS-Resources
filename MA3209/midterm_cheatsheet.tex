\documentclass[frenchspacing,9pt,landscape,a4paper]{article}
\usepackage{amssymb,amsmath,amsthm,amsfonts}
\usepackage{multicol,multirow}
\usepackage{calc}
\usepackage{ifthen}
\usepackage[landscape]{geometry}
\usepackage[colorlinks=true,citecolor=blue,linkcolor=blue]{hyperref}
\usepackage[linewidth=1pt]{mdframed}
\usepackage{enumitem}
\usepackage{times} % Times New Roman font to fit more content

% Even more efficient font
%\usepackage[bitstream-charter]{mathdesign}
\usepackage{charter}
\usepackage[T1]{fontenc}

\usepackage{graphicx}

\setlist[itemize]{align=parleft,left=0pt..1em}
\setlist[enumerate]{align=parleft,left=0pt..1em}

\ifthenelse{\lengthtest { \paperwidth = 11in}}
    { \geometry{top=.5in,left=.5in,right=.5in,bottom=.5in} }
	{\ifthenelse{ \lengthtest{ \paperwidth = 297mm}}
		{\geometry{top=0.8cm,left=0.8cm,right=0.8cm,bottom=0.8cm} }
		{\geometry{top=0.8cm,left=0.8cm,right=0.8cm,bottom=0.8cm} }
	}
\pagestyle{empty}
\makeatletter
\renewcommand{\section}{\@startsection{section}{1}{0mm}%
                                {-1ex plus -.5ex minus -.2ex}%
                                {0.5ex plus .2ex}%x
                                {\normalfont\large\bfseries}}
\renewcommand{\subsection}{\@startsection{subsection}{2}{0mm}%
                                {-1explus -.5ex minus -.2ex}%
                                {0.5ex plus .2ex}%
                                {\normalfont\normalsize\bfseries}}
\renewcommand{\subsubsection}{\@startsection{subsubsection}{3}{0mm}%
                                {-1ex plus -.5ex minus -.2ex}%
                                {1ex plus .2ex}%
                                {\normalfont\small\bfseries}}
\makeatother
\setcounter{secnumdepth}{0}
\setlength{\parindent}{0pt}
\setlength{\parskip}{0pt plus 0.5ex}
\setlength{\columnseprule}{0.1pt}

\newcommand{\BR}{\mathbb R}
\newcommand{\BC}{\mathbb C}
\newcommand{\BF}{\mathbb F}
\newcommand{\BQ}{\mathbb Q}
\newcommand{\BZ}{\mathbb Z}
\newcommand{\BN}{\mathbb N}

\newcommand{\mb}[1]{\mathbf #1}

\newcommand{\floor}[1]{\left\lfloor #1 \right\rfloor}
\newcommand{\ceiling}[1]{\left\lceil #1 \right\rceil}
\newcommand{\abs}[1]{\left\lvert #1 \right\rvert}
\newcommand{\norm}[1]{\left\lVert #1 \right\rVert}
\newcommand{\innerproduct}[2]{\left\langle #1, #2 \right\rangle}

\newcommand{\ddx}{\frac{d}{dx}}
\newcommand{\curl}{\text{curl }}
\newcommand{\dvg}{\text{div }}

\newcommand{\adj}{\text{adj }}

\DeclareMathOperator*{\argmax}{arg\,max}
\DeclareMathOperator*{\argmin}{arg\,min}
\DeclareMathOperator{\pr}{Pr}
\DeclareMathOperator{\var}{Var}
\DeclareMathOperator{\diam}{diam}

\theoremstyle{remark}
\newtheorem*{thm}{\textbf{Theorem}}
\newtheorem*{defn}{\textbf{Definition}}
\newtheorem*{prop}{\textbf{Proposition}}
\newtheorem*{lem}{\textbf{Lemma}}
\newtheorem*{cor}{\textbf{Corollary}}
\newtheorem*{rem}{\textbf{Remark}}

% -----------------------------------------------------------------------
\title{MA3209 cheatsheet}

\begin{document}

\raggedright
\footnotesize

\begin{multicols}{4} % can work with 4 too
\setlength{\premulticols}{1pt}
\setlength{\postmulticols}{1pt}
\setlength{\multicolsep}{1pt}
\setlength{\columnsep}{2pt}
\begin{mdframed}
\begin{center}
    \large{\textbf{MA3209 Metric and Topological Spaces}} \\
    \normalsize{\textbf{AY24/25 Semester 1}}\\
    \small{by Isaac Lai}
\end{center}	
\end{mdframed}
\section{Chapter 1}
\subsection{Topological Spaces}
\begin{defn}
    A \textbf{topology} on a set $X$ is a collection  $\mathcal{T}$ of subsets of  $X$ such that
     \begin{enumerate}
         \item $\emptyset,X\in\mathcal{T}$
         \item  (Closure under arbitrary union) $\{U_\alpha\}_{\alpha\in I}\in\mathcal{T}\implies\bigcup_{\alpha\in
             I}U_\alpha\in\mathcal{T}$
         \item (Closure under finite intersection)  $\{U_1,\dots,U_n\}\in\mathcal{T}\implies\bigcap_{i=1}^n
             U_i\in\mathcal{T}$
     \end{enumerate} $(X,\mathcal{T})$ is a \textbf{topological space}, and  $U\subset X$ is \textbf{open}
     if  $U\in\mathcal{T}$
\end{defn}
\subsubsection{Examples}
Let $X$ be any set.
\begin{itemize}
    \item $\mathcal{T}=\{\emptyset,X\}$ is the \textbf{trivial topology}
    \item  $\mathcal{T}=\{\text{subsets of }X\}$ is the \textbf{discrete topology}
    \item $\mathcal{T}=\{X-U:U\subset X\text{ is finite}\}\cup\{\emptyset\}$ is the \textbf{cofinite
        topology}
\end{itemize}
\begin{defn}
    A \textbf{basis} for a topology of a set $X$ is a collection  $\mathcal{B}$ of  $X$'s subsets such that
     \begin{enumerate}
         \item $\mathcal{B}$ covers  $X$
         \item  $\forall x\in X$ and  $B_1,B_2\in\mathcal{B}$ such that $x\in B_1\cap B_2$, $\exists
             B\in\mathcal{B}$ such that  $x\in B\subset B_1\cap B_2$
     \end{enumerate} The topology generated by $\mathcal{B}$ is  
     \[\mathcal{T}=\{U\subset X:\forall x\in U,\exists B\in\mathcal{B}\text{ such that }x\in B\subset U\}\]
\end{defn}
\subsubsection{Remarks}
\begin{itemize}
    \item If $\mathcal{B}$ is a basis of topology  $\mathcal{T}$, then  $\mathcal{T}$ is the collection of
        all unions of elements in  $\mathcal{B}$
    \item The topology on  $\BR^n$ generated by the open balls is the \textbf{standard topology}
\end{itemize}
\begin{defn}
    Let $X$ be a set and  $\mathcal{T},\mathcal{T}'$ be topologies on  $X$.  $\mathcal{T}$ is
    \textbf{finer} than  $\mathcal{T}'$ (equivalently,  $\mathcal{T}'$ is \textbf{coarser} than
    $\mathcal{T}$) if  $\mathcal{T}'\subset\mathcal{T}$. The topology generated by a basis $\mathcal{B}$ is
    the coarsest topology containing  $\mathcal{B}$.
\end{defn}
\begin{prop}
    Let $\mathcal{B},\mathcal{B}'$ be bases of topologies  $\mathcal{T},\mathcal{T}'$ respectively on  $X$.
    TFAE:
     \begin{enumerate}
         \item $\mathcal{T}'$ is finer than  $\mathcal{T}$
         \item  $\forall B\in\mathcal{B}$,  $\forall x\in B$,  $\exists B'\in\mathcal{B}'$ such that  $x\in
             B'\subset B$
    \end{enumerate}
\end{prop}
\begin{defn}
    A \textbf{subbasis} $\mathcal{S}$ of a set  $X$ is a collection of subsets of  $X$ whose union equals
    $X$. The \textbf{topology generated by $\mathcal{S}$} is a collection $\mathcal{T}$ of all unions of
    finite intersection of sets in  $\mathcal{S}$
\end{defn}
\subsection{Metric spaces}
\begin{defn}
    A \textbf{metric} on a set $X$ is a function  $d:X\times X\to\BR$ such that
     \begin{enumerate}
         \item (Nonnegativity) $d(x,y)\geq 0\ \forall x,y\in X$
         \item (Positive definiteness)  $d(x,y)=0\iff x=y$
         \item (Symmetry)  $d(x,y)=d(y,x)$
         \item (Triangle inequality)  $d(x,y)+d(y,z)\geq d(x,z)\ \forall x,y,z\in X$
     \end{enumerate} $(X,d)$ is a \textbf{metric space}. If only 1, 3, and 4 hold, then $d$ is a
     \textbf{pseudo-metric}. If only 1, 2, and 4 hold,  $d$ is a \textbf{quasi-metric}.
\end{defn}
\begin{defn}
    A \textbf{norm} on a $\mathbb{K}$-vector space  $V$ is a function  $\norm{\cdot}:V\to\BR$ that
    satisfies
     \begin{enumerate}
         \item (Nonnegativity) $\norm{x}\geq 0\ \forall x\in V$ 
         \item (Positive definiteness) $\norm{x}=0\iff x=0$
         \item  (Absolute homogeneity) $\norm{\lambda x}=\abs{\lambda}\norm{x}$
             $\forall\lambda\in\mathbb{K}$ and $\forall x\in V$ 
         \item (Triangle inequality)  $\norm{x+y}\leq\norm{x}+\norm{y}$
    \end{enumerate}
\end{defn}
\subsubsection{Examples}
\begin{itemize}
    \item The \textbf{discrete metric} is
        \begin{align*}
            d(x,y)=
            \begin{cases}
                1, & \text{ if }x\neq y\\
                0, & \text{ if } x=y
            \end{cases}
        \end{align*}
    \item The \textbf{$l_p$-norm} is $V=\mathbb{K}^n,p\geq 1,\norm{x}_p=(\abs{x_1}^p+\cdots+\abs{x_n}^p)^{1 /p},x\in\mathbb{K}^n$ 
    \item The \textbf{$l_\infty$-norm} is
        $V=\mathbb{K}^n,\norm{x}_\infty=\max\{\abs{x_1},\dots,\abs{x_n}\},x\in\mathbb{K}^n$
\end{itemize}
\begin{defn}
    Let $A,B$ be nonempty subsets of a metric space  $(X,d)$.
     \begin{itemize}
         \item The \textbf{distance} between $A$ and  $B$ is  $d(A,B)=\inf\{d(x,y):x\in A,y\in B\}$
         \item The \textbf{diameter} of a set  $A\subset X$ is  $\diam(A)=\sup\{d(x,y):x,y\in A\}$
         \item A set $A\subset X$ is bounded if  $\diam(A)<+\infty$
    \end{itemize}
\end{defn}
\begin{defn}
    \begin{itemize}
        \item The topology on $X$ \textbf{induced} by a metric  $d$ is the topology generated by
            $\mathcal{B}_d$.
        \item A topology  $\mathcal{T}$ on  $X$ is metrizable if there is a metric on  $X$ that induces
            $\mathcal{T}$
    \end{itemize}
\end{defn}
\subsubsection{Remarks}
\begin{itemize}
    \item The discrete metric generates the discrete topology
    \item Every $l_p$-metric on  $\BR^n$ generates the standard topology
\end{itemize}
\subsection{Subspaces of topological spaces}
\begin{defn}
    Let $(Y,\mathcal{T}_Y)$ be a topological space and $X\subset Y$. Then  $\mathcal{T}_X=\{U\cap
    X:U\in\mathcal{T}_Y\}$ is the \textbf{subspace topology} on $X$.
\end{defn}
\begin{defn}
    Let $(Y,\mathcal{T}_Y)$ be a topological space,  $X\subset Y$ be a subset, and  $\mathcal{T}_X$  be the
    subspace topology. Then $X$ is the  \textbf{subspace} of  $Y$ with respect to  $\mathcal{T}_X$.
\end{defn}
\begin{defn}
    Given a subset $A$ of a metric space  $(X,d)$, the \textbf{restriction} of  $d$ to  $A$ is the metric
     $d_A(x,y)=d(x,y)\ \forall x,y\in A$. The topology induced by this metric is the subspace topology.
\end{defn}
\subsubsection{Results}
\begin{itemize}
    \item If $\mathcal{T}_X$ is a topology and  $\mathcal{B}$ is a basis for  $\mathcal{T}_Y$, then
        $\{B\cap X:B\in\mathcal{B}\}$ is a basis for  $\mathcal{T}_X$
    \item If  $X\subset Y$ is open and  $U\subset X$ is open, then  $U\subset Y$ is open
\end{itemize}
\subsection{Closed sets, closure, and limit points}
\begin{defn}
    Let $(X,\mathcal{T})$ be a topological space. A subset  $A\subset X$ is closed if  $X\setminus
    A\in\mathcal{T}$.
\end{defn}
\begin{prop}
    Let $X$ be a topological space.
     \begin{itemize}
         \item If $\{G_\alpha\}_{\alpha\in I}$ is an arbitrary collection of closed sets in  $X$, then
             $\bigcap_{\alpha\in I}G_\alpha\subset X$ is closed.
         \item If  $G_1,\dots,G_n$ are closed sets in $X$, then $\bigcup_{i=1}^n G_i\subset X$ is closed.
         \item If  $Y\subset X$, then  $A\subset Y$ is closed is equivalent to  $A=G\cap Y$ for some closed
              $G\subset X$.
         \item If $Y\subset X$ is closed and  $A\subset Y$ is closed, then  $A\subset X$ is closed.
    \end{itemize}
\end{prop}
\begin{defn}
    Let $(X,\mathcal{T})$ be a topological space and  $A\subset X$.
     \begin{enumerate}
         \item The \textbf{interior} of $A$ is  $\mathring{A}=\bigcup_{U\in\mathcal{T},U\subset A}U$.
         \item The \textbf{closure} of  $A$ is  $\overline{A}=\bigcap_{X\setminus
             G\in\mathcal{T},G\supset A}G$.
         \item The \textbf{boundary} of  $A$ is  $\partial A=\overline{A}-\mathring{A}$.
    \end{enumerate}
\end{defn}
\begin{rem}
    If $\mathring{A}\subset A\subset\overline{A}$, then
     \begin{itemize}
         \item $\mathring{A}=A\iff A$ is open.
         \item  $\overline{A}=A\iff A$ is closed.
    \end{itemize}
\end{rem}
\begin{defn}
    Let $X$ be a topological space and  $A\subset X$. A point  $x\in X$ is a \textbf{limit point} of  $A$
    if every open  $U\subset X$ containing  $x$ intersects  $A\setminus\{x\}$.
\end{defn}
\begin{prop}
    Let $X$ be a topological space,  $A\subset X$.
     \begin{enumerate}
         \item $x\in\overline{A}\iff\forall$ open  $U$ containing  $x$,  $U\cap A\neq\emptyset$.
         \item If  $A'$ is the set of limit points of  $A$, then  $\overline{A}=A\cup A'$.
    \end{enumerate}
\end{prop}
\begin{defn}
    A sequence $(x_1,x_2,x_3,\dots)=\{x_i\}_{i=1}^\infty$ of points in a topological space $X$ converges to
     $x\in X$ if for any neighbourhood  $U$ containing  $x$,  $\exists N>0$ such that  $x_k\in U$ for all 
     $k>N$. This is written as  $x_i\to x$.  $x$ being a limit point of the sequence does not imply that
     $x_i\to x$, and  $x_i\to x$ does not imply that  $x$ is a limit point of the sequence either.
\end{defn}
\subsection{Continuity}
\begin{defn}
    Let $X$ and  $Y$ be topological spaces. A map  $f:X\to Y$ is \textbf{continuous} if for any open set
    $U\subset Y,f^{-1}(U)\subset X$ is open.
\end{defn}
\begin{prop}
    Let $X$ and  $Y$ be topological spaces and  $f:X\to Y$. TFAE:
     \begin{enumerate}
         \item $f$ is continuous
         \item  $\forall A\subset X$,  $f(\overline{A})\subset\overline{f(A)}$
         \item For any closed set  $B\subset Y$,  $f^{-1}(B)\subset X$ is closed
         \item For any  $x\in X$ and any open set  $V\subset Y$ containing  $f(x)$, there exists open
             $U\subset X$ containing  $x$ such that  $f(U)\subset V$
    \end{enumerate}
\end{prop}
\begin{prop}[Pasting Lemma]
    Let $X=A\cup B$ where  $A,B\subset X$ are both closed (or both open). Let  $f:A\to Y$ and  $g:B\to Y$
    be continuous. If  $f(x)=g(x)$ for all  $x\in A\cap B$, then  $h:X\to Y$ defined by
    \begin{align*}
        h(x)=\begin{cases}
            f(x), & \text{ if } x\in A\\
            g(x), & \text{ if } x\in B
        \end{cases}
    \end{align*} is continuous.
\end{prop}
\begin{rem}
    Given a topology $\mathcal{T}_Y$ on  $Y$ and a map  $f:X\to Y$, the \textbf{pull back} topology on  $X$
    is defined as  $\mathcal{T}_X=\{f^{-1}(U):U\in\mathcal{T}_Y\}$. This is the coarsest topology on  $X$
    such that  $f$ is continuous.
\end{rem}
\begin{defn}
    Let $(X,d_X)$ and  $(Y,d_Y)$ be two metric spaces. A map  $f:X\to Y$ is \textbf{uniformly continuous}
    on  $X$ if for any  $\epsilon>0$, there exists  $\delta>0$ such that if  $x,y\in X$ satisfy
    $d_X(x,y)<\delta$, then  $d_Y(f(x),f(y))<\epsilon$.
\end{defn}
\begin{prop}
    Let $(X,d_X)$ and  $(Y,d_Y)$ be metric spaces. A map  $f:X\to Y$ is uniformly continuous iff for any
    two sequences  $\{x_i\}_{i=1}^\infty$ and  $\{y_i\}_{i=1}^\infty$ in  $X$ such that  $d_X(x_i,y_i)\to
    0$, we have  $d_Y(f(x_i),f(y_i))\to 0$.
\end{prop}
\begin{defn}
    Let $f_i:X\to Y$ be a sequence of maps from a set  $X$ to a metric space  $(Y,d)$:
     \begin{itemize}
         \item $\{f_i\}_{i=1}^\infty$ \textbf{converges pointwise} to  $f:X\to Y$ if  $f_i(x)\to f(x)$ for
             any  $x\in X$.
         \item  $\{f_i\}_{i=1}^\infty$ \textbf{converges uniformly} to  $f:X\to Y$ if for any $\epsilon>0$,
             there exists  $N>0$ such that for all  $i\geq N$ and any  $x\in X$,
             $d(f_i(x),f(x))<\epsilon$.
    \end{itemize}
\end{defn}
\subsection{Standard constructions}
\subsubsection{Product of topological spaces}
\begin{defn}
    Let $\{X_\alpha\}_{\alpha\in\Lambda}$ be nonempty sets.
     \begin{itemize}
         \item The \textbf{product} is defined as
             $\prod_{\alpha\in\Lambda}X_\alpha=\{(x_\alpha)_{\alpha\in\Lambda}:x_\alpha\in
             X_\alpha,\forall\alpha\in\Lambda)\}$
         \item For all $\alpha\in\Lambda$, the map  $\pi_{X_\alpha}:\prod_{\alpha\in\Lambda}X_\alpha\to
             X_\alpha$ defined by  $(x_\alpha)_{\alpha\in\Lambda}\mapsto x_\alpha$ is the
             \textbf{projection} to the  $\alpha$-th factor.
    \end{itemize}
\end{defn}
\begin{defn}
    \begin{itemize}
        \item If $(X_\alpha,\mathcal{T}_\alpha)_{\alpha\in\Lambda}$ are topological spaces, the
            \textbf{product topology} on  $\prod_{\alpha\in\Lambda}X_\alpha$ is the topology generated by
            the subbasis
            $\mathcal{S}=\{\pi_{X_\alpha}^{-1}(U_\alpha):\alpha\in\Lambda,U_\alpha\in\mathcal{T}_\alpha\}$.
        \item If  $(X_\alpha,\mathcal{T}_\alpha)_{\alpha\in\Lambda}$ are topological spaces, the
            \textbf{box topology} on  $(X_\alpha,\mathcal{T}_\alpha)_{\alpha\in\Lambda}$ is the topology
            generated by the basis  $\mathcal{B}=\{\prod_{\alpha\in\Lambda}U_\alpha:U_\alpha\subset
            X_\alpha\text{ is open}\}$.
    \end{itemize} The product and box topologies are the \textbf{same for finite} product but
    \textbf{different for infinite} product.
\end{defn}
\begin{prop}
    Let $\{X_{\alpha}\}_{\alpha\in\Lambda}$ be topological spaces. For any  $\alpha\in\Lambda$, let
    $\pi_{X_\alpha}:\prod_{\alpha\in\Lambda}X_\alpha\to X_\alpha$ be the projection to the  $\alpha$-th
    factor:
     \begin{enumerate}
         \item The product topology on $\prod_{\alpha\in\Lambda}X_\alpha$ is the coarsest topology such
             that  $\pi_{X_\alpha}$ is continuous for any  $\alpha\in\Lambda$.
         \item Let  $Y$ be a topological space, and for any  $\alpha\in\Lambda$, let  $f_\alpha:Y\to
             X_\alpha$. The map  $f=\prod_{\alpha\in\Lambda} f_\alpha:Y\to\prod_{\alpha\in\Lambda}X_\alpha$
             defined by  $y\mapsto(f_\alpha(y))_{\alpha\in\Lambda}$ is continuous iff  $f_\alpha$ is
             continuous for every  $\alpha\in\Lambda$.
    \end{enumerate}
\end{prop}
\begin{prop}
    Let $X$ be a topological space and  $f,g:X\to\BR$ be continuous. Then  $f+g$,  $f-g$, and  $f\cdot g$
    are continuous. Also, if  $0\notin g(X)$, then  $\frac{f}{g}$ is continuous.
\end{prop}
\subsubsection{Products of metric spaces}
\begin{defn}
    If $(X_1,d_{X_1}),\dots,(X_n,d_{X_n})$ are metric spaces, there are two common metrics on
    $X_1\times\cdots\times X_n$:
    \begin{align*}
        d_1((x_1,\dots,x_n),(y_1,\dots,y_n))&=\sum_{i=1}^n d_{X_i}(x_i,y_i)\\
        d_\infty((x_1,\dots,x_n),(y_1,\dots,y_n))&=\max_{i=1,\dots,n}(d_{X_i}(x_i,y_i)) 
    \end{align*}
\end{defn}
\begin{rem}
    \begin{itemize}
        \item If $\mathcal{B}_1,\dots,\mathcal{B}_n$ are bases for the topological spaces
            $(X_1,\mathcal{T}_1),\dots,(X_n,\mathcal{T}_n)$ respectively, then
            $\mathcal{B}_1\times\cdots\times\mathcal{B}_n$ is a basis for  $X_1\times\cdots\times X_n$
            that generates the product topology.
        \item If $(X_1,d_{X_1}),\dots,(X_n,d_{X_n})$ are metric spaces that induce topologies
            $\mathcal{T}_1,\dots,\mathcal{T}_n$ on  $X_1,\dots,X_n$ respectively, then the metrics $d_1
            $and $d_\infty$ on  $X_1\times\cdots\times X_n$ both induce the product topology.
        \item In the infinite product case, let $(X_i,d_{X_i})_{i=1}^\infty$ be metric spaces. Given the
            metric  $d_\infty$ above, we define  $d_\infty:\prod_{i=1}^\infty X_i\times\prod_{i=1}^\infty
            X_i\to\BR$ by  $d_\infty(x,y)=\sup\{d_{X_i}(x_i,y_i):i\in\BZ^+\}$. But this is not well-defined
            as  $d_{X_i}(x_i,y_i)$ might be unbounded as  $i\to\infty$.
    \end{itemize}
\end{rem}
\begin{prop}
    Let $(X,d)$ be a metric space. Then  $\rho:X\times X\to\BR$ given by
    $\rho(x,y)=\frac{d(x,y)}{1+d(x,y)}$ is a metric and its diameter is less than $1$. Furthermore  $\rho$
    and  $d$ induce the same topology on  $X$.
\end{prop}
\subsubsection{Quotient of topological spaces}
\begin{defn}
    Let $X$ and  $Y$ be topological spaces.
     \begin{itemize}
         \item A surjective map $p:X\to Y$ is a \textbf{quotient map} if  $V\subset Y$ is open  $\iff
             p^{-1}(V)\subset X$ is open.
         \item A continuous map  $f:X\to Y$ is \textbf{open (closed)} if  $f(U)$ is open (closed) for any open
             (closed)  $U\subset X$.
    \end{itemize} If a surjective continuous map is open or closed, then it is a quotient map. The
    composition of quotient maps is also a quotient map.
\end{defn}
\begin{defn}
    Let $f:X\to Y$ be a surjective continuous map and  $A\subset X$. THen  $A$ is a \textbf{saturated set}
    wrt  $f$ if  $A=f^{-1}(S)$ for some  $S\subset Y$. Equivalently  $A=f^{-1}(f(A))$.
\end{defn}
\begin{defn}
    Let $f:X\to Y$ be a surjective continuous map.
     \begin{enumerate}
         \item $f$ is a quotient map  $\iff f$ sends every saturated (wrt $f$) open (closed) set to an open
             (closed) set.
         \item If $f$ is a quotient map and  $A\subset X$ is saturated and open (closed), then
             $f|_A:A\to f(A)$ is also a quotient map.
    \end{enumerate}
\end{defn}
\begin{prop}
    If $X$ is a topological space,  $A\subset X$ and  $p:X\to A$ is surjective, then  $\exists!$ topology
    on  $A$ (called the \textbf{quotient topology}) such that  $p$ is a quotient map.
\end{prop}
\begin{defn}
    Let $X$ be a topological space and let  $X^*$ be the cells of a partition of  $X$. Let  $p:X\to X^*$ be
    the surjective map that sends each point in  $X$ to the subset that contains it.  $X^*$ equipped
    with the quotient topology induced by  $p$ is a \textbf{quotient space} of $X$.
\end{defn}
\section{Chapter 2}
\subsection{$T_1$ and $T_2$ spaces}
\begin{defn}
    Let $X$ be a topological space.
     \begin{itemize}
         \item $X$ is  $T_1$ if for any distinct $x,y\in X$, there exists an open set  $U\subset X$ such
             that  $x\in U$ but  $y\notin U$.
         \item  $X$ is  $T_2$ or \textbf{Hausdorff} if for any distinct $x,y\in X$, there exist open
             neighbourhoods  $U,V$ of  $x,y$ respectively such that they are disjoint.
    \end{itemize} 
\end{defn}
\subsubsection{Examples}
\begin{itemize}
    \item Any Hausdorff space is $T_1$
    \item Any metric space is Hausdorff
    \item If $\abs{X}\geq 2$, then the trivial topology is not  $T_1$
    \item The discrete topology is Hausdorff
    \item The cofinite topology is $T_1$. The cofinite topology is Hausdorff iff $X$ is finite 
    \item If $X$ is infinite, then the cofinite topology on  $X$ is not metrizable
\end{itemize}
\begin{prop}
    $X$ is  $T_1\iff\forall x\in X$, $\{x\}$ is closed. It follows that finite sets in metric spaces are
    closed.
\end{prop}
\subsection{First countable space}
\begin{defn}
    Let $X$ be a topological space.
     \begin{itemize}
         \item $\forall x\in X$, a \textbf{countable basis of $X$ at  $x$} is a countable collection
             $\mathcal{B}$ of open sets in  $X$ that contain  $x$ such that every open set in  $X$ that
             contains  $x$ also contains some  $B\in\mathcal{B}$.
         \item  $X$ is \textbf{first countable} if there is a countable basis of  $X$ at  $x$ for every
             $x\in X$.
    \end{itemize}
\end{defn}
\begin{prop}
    Let $X$ be a topological space.
     \begin{enumerate}
         \item Let $A\subset X$. If there exists a sequence  $(x_i)_{i=1}^\infty\subset A$ such that
             $x_i\to x$ as  $i\to\infty$, then  $x\in\overline{A}$. The converse is true if  $X$ is first
             countable.
         \item Let  $f:X\to Y$. If  $f$ is continuous, then for any sequence  $(x_i)_{i=1}^\infty\subset X$
             such that  $x_i\to x$ as  $n\to\infty$, we have  $f(x_i)\to f(x)$ as  $i\to\infty$. The
             converse holds if  $X$ is first countable.
    \end{enumerate}
\end{prop}
\subsection{Compactness}
\begin{defn}
    Let $X$ be a topological space.
     \begin{itemize}
         \item An \textbf{open cover} of $X$ is a collection of open sets  $\{U_\alpha\}_{\alpha\in\Lambda}$ in  $X$
             such that  $\bigcup_{\alpha\in\Lambda}U_\alpha=X$. 
         \item $X$ is \textbf{compact} if every open cover of  $X$ admits a finite subcover.
    \end{itemize}
\end{defn}
\begin{rem}
    $Y\subset X$ is a compact subspace  $\iff$ every collection  $\mathcal{U}$ of open sets in  $Y$ such
    that $Y\subset\bigcup_{U\in\mathcal{U}}$ admits a finite sub-collection
    $\mathcal{U}'\subset\mathcal{U}$ such that  $Y\subset\bigcup_{U\in\mathcal{U}'}U$.
\end{rem}
\begin{prop}
    Every closed subspace of a compact space is compact.
\end{prop}
\begin{prop}
    Every compact subspace of a Hausdorff space is closed.
\end{prop}
\begin{prop}[Tube lemma]
    Let $X$ be a topological space and  $Y$ be a compact topological space. If  $N\subset X\times Y$ is an
    open set that contains  $\{(x_0,y):y\in Y\}$, then $N$ contains $W\times Y$ for some  $W\subset X$ that
    contains $x_0$.
\end{prop}
\begin{cor}
    If $X$ and  $Y$ are compact topological spaces, then  $X\times Y$ is compact.
\end{cor}
\begin{defn}
    A collection $\mathcal{G}$ of subsets of  $X$ has the \textbf{finite intersection property} if every
    finite sub-collection  $\{G_1,\dots,G_n\}\subset\mathcal{G}$ satisfies $\bigcap_{i=1}^n
    G_i\neq\emptyset$.   
\end{defn}
\begin{prop}
    A topological space $X$ being compact is equivalent to  $X$ having the following property: Let
    $\mathcal{G}$ be a collection of closed sets in  $X$. If  $\mathcal{G}$ has the finite intersection
    property, then  $\bigcap_{G\in\mathcal{G}} G\neq\emptyset$.
\end{prop}
\begin{cor}
    If $X$ is compact and  $\{G_i\}_{i=1}^\infty$ is a nested (i.e. $G_{i+1}\subset G_i$ for all
    $i\in\BZ^+$) sequence of closed subsets in $X$, then  $\bigcap_{i=1}^\infty G_i\neq\emptyset$.
\end{cor}
\begin{defn}
    A point $x$ in a topological space is \textbf{isolated} if $\{x\}$ is open in  $X$.
\end{defn}
\begin{thm}
    Let $X$ be a non-empty, compact, Hausdorff space. If  $X$ has no isolated points, then  $X$ is
    uncountable.
\end{thm}
\subsection{Limit points, sequential compactness, and the Lebesgue number}
\begin{defn}
    A topological space $X$ is \textbf{limit point compact} if every infinite subset of  $X$ has a limit
    point in  $X$.
\end{defn}
\begin{prop}
    If $X$ is compact, then it is limit point compact.
\end{prop}
\begin{defn}
    Let $X$ be a topological space.  $X$ is \textbf{sequentially compact} if every sequence in  $X$ has a
    convergent subsequence.
\end{defn}
\begin{defn}
    Let $X$ be a metric space, and let  $\mathcal{U}$ be an open cover of  $X$. A number  $\delta>0$ is a
    \textbf{Lebesgue number} for $\mathcal{U}$ if for all subsets  $S\subset X$ such that
    $\diam(S)<\delta$, there exists  $U\in\mathcal{U}$ such that  $S\subset U$.
\end{defn}
\begin{lem}
    If $X$ is a sequentially compact metric space, then every open cover of $X$ has a Lebesgue number.
\end{lem}
\begin{defn}
    A metric space $X$ is \textbf{totally bounded} if for all $\epsilon>0$, there exists a finite cover of
     $X$ by balls of radius  $\epsilon$.
\end{defn}
\begin{lem}
    If $X$ is sequentially compact and metrizable, then  $X$ is totally bounded.
\end{lem}
\begin{thm}
    If $X$ is metrizable, then TFAE:
     \begin{enumerate}
         \item $X$ is compact.
         \item  $X$ is limit point compact.
         \item  $X$ is sequentially compact.
    \end{enumerate}
\end{thm}
\begin{cor}
    Let $f:(X,d_X)\to(Y,d_Y)$ be continuous. If  $X$ is compact, then  $f$ is uniformly continuous.
\end{cor}

\end{multicols}
\end{document}
