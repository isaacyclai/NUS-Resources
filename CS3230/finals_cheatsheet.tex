\documentclass[11pt]{article}

\usepackage{amsfonts,latexsym,amsthm,amssymb,amsmath,amscd,euscript}
\usepackage{enumerate}
\usepackage{eqparbox}
\usepackage{framed}
\usepackage{fullpage}
\usepackage{hyperref}
    \hypersetup{colorlinks=true,citecolor=blue,urlcolor =black,linkbordercolor={1 0 0}}
\usepackage{blindtext}
\usepackage{bm}
\usepackage{asymptote}
\usepackage{mathtools}

\usepackage[dvipsnames]{xcolor}
\usepackage[framemethod=TikZ]{mdframed}
\usepackage[most]{tcolorbox}

\usepackage{minted}

\theoremstyle{remark}
\newtheorem*{remark}{Remark}
\newtheorem*{notation}{Notation}
\newtheorem*{note}{Note}

\newcommand{\BR}{\mathbb R}
\newcommand{\BC}{\mathbb C}
\newcommand{\BF}{\mathbb F}
\newcommand{\BQ}{\mathbb Q}
\newcommand{\BZ}{\mathbb Z}
\newcommand{\BN}{\mathbb N}

\newcommand{\mb}[1]{\mathbf #1}

\newcommand{\floor}[1]{\left\lfloor #1 \right\rfloor}
\newcommand{\ceiling}[1]{\left\lceil #1 \right\rceil}
\newcommand{\abs}[1]{\left\lvert #1 \right\rvert}
\newcommand{\norm}[1]{\left\lVert #1 \right\rVert}
\newcommand{\innerproduct}[2]{\left\langle #1, #2 \right\rangle}

\newcommand{\ddx}{\frac{d}{dx}}
\newcommand{\curl}{\text{curl }}
\newcommand{\dvg}{\text{div }}

\newcommand{\adj}{\text{adj }}

\DeclareMathOperator*{\argmax}{arg\,max}
\DeclareMathOperator*{\argmin}{arg\,min}

\title{CS3230 final reference}
\date{\today}
\author{Isaac Lai}

\begin{document}
\maketitle
\section{Asymptotic facts}
\begin{align*}
    e^x&\geq 1+x\\
    a^{\log_b c}&=c^{\log_b a}\\
    n!&=\sqrt{2\pi n}\left(\frac{n}{3}\right)^n\left(1+\Theta\left(\frac{1}{n}\right)\right)\text{
    (Stirling's approximation)}\\
    \log(n!)&=\Theta(n\log n)\\
    \sum_{k=0}^n ar^k&= \frac{a(r^{n+1}-1)}{r-1}\text{ (Geometric series)}\\
    \sum_{k=1}^n \frac{1}{k}&=\ln n+O(1)\text{ (Harmonic series)}\\
    \lim_{x\to\infty}\frac{f(x)}{g(x)}&=\lim_{x\to\infty}\frac{f'(x)}{g'(x)}\text{ (L'Hopital's Rule)}\\
\end{align*}
\section{Asymptotic analysis}
\begin{itemize}
    \item $f(n)=O(g(n))$ if  $\exists c>0,n_0>0$ such that $\forall n\geq n_0, 0\leq f(n)\leq cg(n)$
    \item  $f(n)=\Omega(g(n))$ if  $\exists c>0,n_0>0$ such that $\forall n\geq n_0, 0\leq cg(n)\leq f(n)$
    \item  $f(n)=\Theta(g(n))$ if $\exists c_1,c_2>0,n_0>0$ such that $\forall n\geq n_0, 0\leq c_1g(n)\leq
        f(n)\leq c_2g(n)$ i.e. $\Theta(g)=O(g)\cap\Omega(g)$ 
    \item $\lim\limits_{n\to\infty}\frac{f(n)}{g(n)}=0\implies f(n)=o(g(n))$ 
    \item $\lim\limits_{n\to\infty}\frac{f(n)}{g(n)}<\infty\implies f(n)=O(g(n))$    
    \item $0<\lim\limits_{n\to\infty}\frac{f(n)}{g(n)}<\infty\implies f(n)=\Theta(g(n))$
    \item $\lim\limits_{n\to\infty}\frac{f(n)}{g(n)}>0\implies f(n)=\Omega(g(n))$  
    \item $\lim\limits_{n\to\infty}\frac{f(n)}{g(n)}=\infty\implies f(n)=\omega(g(n))$   
\end{itemize}
\section{Recurrences}
General form: $T(n)=aT(\frac{n}{b})+f(n)$
\begin{itemize}
    \item Telescoping: express in form $\frac{T(n)}{g(n)}=\frac{T(\frac{n}{b})}{g(\frac{n}{b})}+h(n)$ 
    \item Recursion tree: draw tree, sum each node (can sum over level first then over height)
    \item \textbf{Master Theorem}: $a\geq 1,b>1,f$ asymptotically positive
        \begin{enumerate}
            \item $f(n)=O(n^{\log_b a-\epsilon})$ for some  $\epsilon>0$.  $T(n)=\Theta(n^{\log_b a})$
            \item  $f(n)=\Theta(n^{\log_b a}\log^k n)$ for some  $k\geq 0$.  $T(n)=\Theta(n^{\log_b
                a}\log^{k+1}n)$
            \item  $f(n)=\Omega(n^{\log_b a+\epsilon})$ for some  $\epsilon>0$ and satisfies
                \textbf{regularity condition}  $af(\frac{n}{b})\leq cf(n)$ for some $c<1$.
                $T(n)=\Theta(f(n))$
        \end{enumerate}
    \item Substitution: guess and check by induction. Induction hypothesis: $c_1n^k-\text{(lower order
        terms)}$
\end{itemize}
\section{Divide and Conquer}
Invariant: condition which is true at the start of every iteration. To show correctness check
\begin{itemize}
    \item Initialisation: invariant is true at iteration 1
    \item Maintenance: if invariant is true for iteration $n$, it remains true for iteration $n+1$ 
    \item Termination: when the algorithm ends, the invariant helps the proof of correctness
\end{itemize}
Strassen's algorithm: matrix multiplication in $O(n^{\log_2 7})$ (splitting into 7 multiplications instead
of 8)

\section{Sorting}
Lower bound for \textbf{comparison-based} sort is $O(n\log n)$ by argument from decision tree
\begin{center}
    \begin{tabular}{l|c|c}
        No. of comparisons & Mergesort & Quicksort\\
        \hline
        Average case & $\Theta(n\log_2 n)$ & $\Theta(n\log_2n)$\\
        Best case & $\Theta(n\log_2n)$ & $\Theta(n\log_2n)$\\
        Worst case & $\Theta(n\log_2n)$ & $\Theta(n^2)$
    \end{tabular}
\end{center}

\section{Randomisation}
\begin{itemize}
    \item Las Vegas algorithms: (1) output always correct (2) running time depends on random bits, with
        small probability that it may be large (3) expected running time is bounded by given time-bound
        function
    \item Monte Carlo algorithms: (1) answer may be incorrect with small probability (2) running time is always bounded by given time-bound function
    \item Union bound: $P(A\cup B)\leq P(A)+P(B)$
    \item Linearity of expectation:  $E[X+Y]=E[X]+E[Y]$ even if  $X$ and  $Y$ are not independent random
        variables
\end{itemize}
\section{Dynamic Programming}
\begin{itemize}
    \item Optimal substructure: optimal solution contains optimal solutions to subproblems
    \item Overlapping subproblems: recursive solution contains a small number of distinct subproblems
        repeated many times
    \item Top-down saves computation of unnecessary subproblems but can suffer from
        overhead of recursive calls, bottom-up is the opposite
\end{itemize}

\section{Greedy}
\begin{itemize}
    \item Greedy choice: pick the largest/smallest/etc. (some extreme value). Show there is always an
        optimal solution to the original problem that makes the greedy choice
    \item Use optimal substructure to show we can combine an optimal solution to the subproblem with the
        greedy choice to get an optimal solution to the original problem
\end{itemize}

\section{Amortised analysis}
\begin{itemize}
    \item Shows that the average cost per operation is small, even though a single operation within the
        sequence might be expensive
    \item Accounting method: charge $i$th operation an amortised cost  $c(i)$. Must ensure the sum of true
        costs  $\sum_{i=1}^nt(i)\leq\sum_{i=1}^nc(i)$ 
    \item Potential method: $\phi(0)=0$,  $\phi(i)\geq 0$ for all  $i$. Amortised cost of  $i$th operation
        $=t(i)+(\phi(i)-\phi(i-1))$. Heuristic: try to find some quantity that is "decreasing" during the
        operation
\end{itemize}

\section{Reductions and NP-completeness}
\begin{itemize}
    \item Poly-time: polynomial in terms of the length of the encoding of the problem instance
    \item Pseudo-polynomial: polynomial in numeric value of the input, but not necessarily in length of the
        input
    \item $A\leq_p B$: $A$ is poly-time reducible to  $B$. So if  $B$ has a poly-time algorithm, so does
        $A$. Conversely, $A$ can be seen as a special case of  $B$, so if  $A$ is hard, then  $B$ is hard
        too
    \item Decision reduces to optimisation (given value of optimal solution, simply check if it is  $\leq
        k$)
    \item NP-complete: problem must be both in NP and NP-hard
    \item NP-complete problems: Circuit SAT, CNF-SAT, 3-SAT,
        MAX-2-SAT, Vertex Cover, Independent Set, Max-Clique,
        (Directed/Undirected) Hamiltonian Cycle,
        Travelling Salesperson, Parition, Subset Sum, 0-1 Knapsack
\end{itemize}

\section{Order statistics}
\begin{itemize}
    \item Worst case linear time algorithm to select the rank-$i$ element
\end{itemize}
\end{document}
